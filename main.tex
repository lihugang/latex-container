
\documentclass[12pt, a4paper, addpoints]{exam}
\usepackage{xeCJK}
\setCJKmainfont{SimSun}[BoldFont=SimHei,ItalicFont=KaiTi]

\footer{}{第 \thepage 页 (共 \pageref{LastPage} 页)}{}
\firstpageheader{古诗文名言名句选读 默写 / 2023古诗文}{}{姓名:\qquad\qquad}
\runningheader{}{古诗文名言名句选读 默写 / 2023古诗文}{}

\usepackage{multicol}
\usepackage{lastpage}

\usepackage[
pdfa=true,
unicode=true,
hidelinks,
pdfauthor={李沪纲},
pdftitle={古诗文名言名句选读 默写 / 2023古诗文},
pdfsubject={古诗文名言名句选读 默写 / 2023古诗文}]{hyperref}

\usepackage{geometry}
\geometry{a4paper,scale=0.8}

\usepackage{ulem}

% \usepackage{fancyhdr}
% \pagestyle{fancy}
% \fancyhead[L]{古诗文名言名句选读 默写 / 2023古诗文}
% \fancyhead[R]{班级:\qquad 姓名:\qquad\qquad}
% \fancyfoot[C]{第 \thepage 页 (共 \pageref{LastPage} 页)}

\pointformat{(\thepoints)}
\pointname{分}

\setlength{\parindent}{2em}

\begin{document}

\pagestyle{headandfoot}

\begin{center}
\fbox{\fbox{\parbox{5.5in}{\centering
古诗文名言名句选读 默写 / 2023古诗文

命题人:李沪纲

(答题时间 10 分钟,共 40 题,满分 80 分)

于 2023/7/17 20:30:47 导出}}}
\end{center}
\vspace{5mm}

\normalsize
\vspace{5mm}

\section{\normalsize{填空题 (本大题共 40 题,满分 80 分)}}
\hspace{1.5cm}
\begin{multicols}{2}
\begin{questions}
\question[2] 近水救不得\uline{\qquad\qquad\qquad}

\question[2] \uline{\qquad\qquad\qquad},饥寒不累其心

\question[2] 迅雷\uline{\qquad\qquad\qquad}掩耳

\question[2] 桃李\uline{\qquad\qquad\qquad}

\question[2] \uline{\qquad\qquad\qquad}必从勤苦得

\question[2] 劳于读书,\uline{\qquad\qquad\qquad}

\question[2] \uline{\qquad\qquad\qquad},同气相求

\question[2] 君子患无德,\uline{\qquad\qquad\qquad}

\question[2] \uline{\qquad\qquad\qquad}到晓温

\question[2] 奇文共欣赏,\uline{\qquad\qquad\qquad}

\question[2] 君子坦荡荡,\uline{\qquad\qquad\qquad}

\question[2] 富贵\uline{\qquad\qquad\qquad}勤苦得

\question[2] 飘风不终朝,\uline{\qquad\qquad\qquad}

\question[2] \uline{\qquad\qquad\qquad},自有留人处

\question[2] 一波才动\uline{\qquad\qquad\qquad}

\question[2] 此处不留人,\uline{\qquad\qquad\qquad}

\question[2] \uline{\qquad\qquad\qquad},如见其人

\question[2] 剑老无芒,\uline{\qquad\qquad\qquad}

\question[2] 蜗牛角上\uline{\qquad\qquad\qquad}

\question[2] 当局者迷,\uline{\qquad\qquad\qquad}

\question[2] \uline{\qquad\qquad\qquad},逸于作文

\question[2] 江山易改,\uline{\qquad\qquad\qquad}

\question[2] 防民之口,\uline{\qquad\qquad\qquad}

\question[2] 大夏将倾,\uline{\qquad\qquad\qquad}

\question[2] \uline{\qquad\qquad\qquad}不及掩耳

\question[2] \uline{\qquad\qquad\qquad},旁观者清

\question[2] \uline{\qquad\qquad\qquad},禀性难移

\question[2] 既来之,\uline{\qquad\qquad\qquad}

\question[2] \uline{\qquad\qquad\qquad},非一木可支

\question[2] \uline{\qquad\qquad\qquad},俯不怍于人

\question[2] \uline{\qquad\qquad\qquad},专欲难成

\question[2] \uline{\qquad\qquad\qquad},谬以千里

\question[2] \uline{\qquad\qquad\qquad},骤雨不终日

\question[2] 毁誉不干其守,\uline{\qquad\qquad\qquad}

\question[2] 一语天然\uline{\qquad\qquad\qquad}

\question[2] \uline{\qquad\qquad\qquad},小人长戚戚

\question[2] \uline{\qquad\qquad\qquad}淡如水

\question[2] \uline{\qquad\qquad\qquad}满天下

\question[2] \uline{\qquad\qquad\qquad}偏逢连夜雨

\question[2] 富贵必从\uline{\qquad\qquad\qquad}

\end{questions}
\end{multicols}

\end{document}

