
\documentclass[12pt, a4paper, addpoints]{exam}
\usepackage{xeCJK}

\footer{}{第 \thepage 页 (共 \pageref{LastPage} 页)}{}
\firstpageheader{诗句默写 / 2023古诗文}{}{姓名:\qquad\qquad}
\runningheader{}{诗句默写 / 2023古诗文}{}

\usepackage{lastpage}

\usepackage[
pdfa=true,
unicode=true,
hidelinks,
pdfauthor={李沪纲},
pdftitle={诗句默写 / 2023古诗文},
pdfsubject={诗句默写 / 2023古诗文}]{hyperref}

\usepackage{geometry}
\geometry{a4paper,scale=0.8}

\usepackage{ulem}

% \usepackage{fancyhdr}
% \pagestyle{fancy}
% \fancyhead[L]{诗句默写 / 2023古诗文}
% \fancyhead[R]{班级:\qquad 姓名:\qquad\qquad}
% \fancyfoot[C]{第 \thepage 页 (共 \pageref{LastPage} 页)}

\pointformat{(\thepoints)}
\pointname{分}

\setlength{\parindent}{2em}

\begin{document}

\pagestyle{headandfoot}

\begin{center}
\fbox{\fbox{\parbox{5.5in}{\centering
诗句默写 / 2023古诗文

命题人:李沪纲

(答题时间 10 分钟,共 20 题,满分 40 分)

于 2023/7/17 10:11:23 导出}}}
\end{center}
\vspace{5mm}

\normalsize
\vspace{5mm}

\section{\normalsize{填空题 (本大题共 20 题,满分 40 分)}}
\hspace{1.5cm}
\begin{questions}
\question[2] \uline{\qquad\qquad\qquad\qquad},北风吹起数声雷。

\question[2] 渡远荆门外,\uline{\qquad\qquad\qquad\qquad}。

\question[2] 渡头馀落日,\uline{\qquad\qquad\qquad\qquad}。

\question[2] 缺月挂疏桐,\uline{\qquad\qquad\qquad\qquad}。

\question[2] \uline{\qquad\qquad\qquad\qquad},阴晴众壑殊。

\question[2] \uline{\qquad\qquad\qquad\qquad},斜光到晓穿朱户。

\question[2] \uline{\qquad\qquad\qquad\qquad},临风听暮蝉。

\question[2] 相顾无相识,\uline{\qquad\qquad\qquad\qquad}。

\question[2] \uline{\qquad\qquad\qquad\qquad},潮退渔船阁岸斜。

\question[2] 仍怜故乡水,\uline{\qquad\qquad\qquad\qquad}。

\question[2] 东皋薄暮望,\uline{\qquad\qquad\qquad\qquad}。

\question[2] \uline{\qquad\qquad\qquad\qquad},端居耻圣明。

\question[2] \uline{\qquad\qquad\qquad\qquad},金陵王气黯然收。

\question[2] \uline{\qquad\qquad\qquad\qquad},我辈复登临。

\question[2] 分野中峰变,\uline{\qquad\qquad\qquad\qquad}。

\question[2] 水落鱼梁浅,\uline{\qquad\qquad\qquad\qquad}。

\question[2] 郭边万户皆临水,\uline{\qquad\qquad\qquad\qquad}。

\question[2] \uline{\qquad\qquad\qquad\qquad},江入大荒流。

\question[2] \uline{\qquad\qquad\qquad\qquad},徒有羡鱼情。

\question[2] 白鸟一双临水立,\uline{\qquad\qquad\qquad\qquad}。

\end{questions}

\end{document}

