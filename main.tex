
\documentclass[12pt, a4paper, addpoints, answers]{exam}
\printanswers

\usepackage{xeCJK}
\setCJKmainfont{SimSun}[BoldFont=SimHei,ItalicFont=KaiTi]

\footer{}{第 \thepage 页 (共 \pageref{LastPage} 页)}{}
\firstpageheader{Test}{}{\textbf{标准答案}}
\runningheader{}{Test}{\textbf{标准答案}}

\usepackage{amssymb}
\usepackage{multicol}
\usepackage{lastpage}

\usepackage{xcolor}

\usepackage[
pdfa=true,
unicode=true,
hidelinks,
pdfauthor={李沪纲},
pdftitle={Test},
pdfsubject={Test}]{hyperref}

\usepackage{geometry}
\geometry{a4paper,scale=0.8}

\usepackage{ulem}

\pointformat{(\thepoints)}
\pointname{分}
\checkboxchar{$\square$}
\checkedchar{$\blacksquare$}

\setlength{\parindent}{2em}

\begin{document}

\pagestyle{headandfoot}

\begin{center}
\fbox{\fbox{\parbox{5.5in}{\centering
Test

\textbf{标准答案}

命题人:李沪纲

(答题时间 2 分钟,共 \numquestions 题,满分 \numpoints 分)

于 2023/7/25 12:58:52 导出}}}
\end{center}
\vspace{5mm}

\normalsize
\vspace{5mm}

\section{\normalsize{填空题 (本大题共 20 题,满分 60 分)}}
\hspace{1.5cm}

\begin{questions}
\question[3] 双飞日月驱神骏,\fillin[半缺河山待女娲] 。

\question[3] \fillin[夕阳依旧垒] ,寒磬满空林。

\question[3] 王濬楼船下益州,\fillin[金陵王气黯然收] 。

\question[3] \fillin[三顾频烦天下计] ,两朝开济老臣心。

\question[3] \fillin[白云回望合] ,青霭入看无。

\question[3] \fillin[浮云一别后] ,流水十年间。

\question[3] \fillin[平沙日未没] ,黯黯见临洮。

\question[3] \fillin[野寺来人少] ,云峰隔水深。

\question[3] 霜后芙蓉落远洲,\fillin[雁行初过客登楼] 。

\question[3] \fillin[欲投人处宿] ,隔水问樵夫。

\question[3] \fillin[黄梅时节家家雨] ,青草池塘处处蛙。

\question[3] 玉颜自古为身累,\fillin[肉食何人与国谋] 。

\question[3] 饮马渡秋水,\fillin[水寒风似刀] 。

\question[3] 明朝挂帆去,\fillin[枫叶落纷纷] 。

\question[3] 水落鱼梁浅,\fillin[天寒梦泽深] 。

\question[3] 太乙近天都,\fillin[连山接海隅] 。

\question[3] 倚杖柴门外,\fillin[临风听暮蝉] 。

\question[3] 余亦能高咏,\fillin[斯人不可闻] 。

\question[3] 宝马雕车香满路。凤箫声动,\fillin[玉壶光转] ,一夜鱼龙舞。

\question[3] \fillin[岂有千山与万山] ,别离何易来何难。

\end{questions}

\hspace{5cm}

\section{\normalsize{填空题 (本大题共 10 题,满分 20 分)}}
\hspace{1.5cm}
\begin{multicols}{2}
\begin{questions}
\question[2] \fillin[一波才动] 万波随

\question[2] 当局者迷,\fillin[旁观者清] 

\question[2] \fillin[行百里者] 半九十

\question[2] \fillin[飘风不终朝] ,骤雨不终日

\question[2] \fillin[防民之口] ,甚于防川

\question[2] \fillin[一步一陟一回顾] ,我脚高时他更高

\question[2] 一语天然\fillin[万古新] 

\question[2] 近水\fillin[救不得] 远火

\question[2] \fillin[既来之] ,则安之

\question[2] \fillin[先国家之急] 而后私仇

\end{questions}
\end{multicols}

\hspace{5cm}

\section{\normalsize{多选题 (本大题共 4 题,满分 20 分)}}
\hspace{1.5cm}

\begin{questions}
\question[5] AB

\begin{oneparcheckboxes}
\CorrectChoice 1
\CorrectChoice 2
\choice  3
\choice  4
\end{oneparcheckboxes}

\question[5] ABCD

\begin{oneparcheckboxes}
\CorrectChoice 1
\CorrectChoice 2
\CorrectChoice 3
\CorrectChoice 4
\end{oneparcheckboxes}

\question[5] C

\begin{oneparcheckboxes}
\choice  1
\choice  2
\CorrectChoice 3
\choice  4
\end{oneparcheckboxes}

\question[5] none

\begin{oneparcheckboxes}
\choice  1
\choice  2
\choice  3
\choice  4
\end{oneparcheckboxes}

\end{questions}

\hspace{5cm}

\section{\normalsize{单选题 (本大题共 1 题,满分 20 分)}}
\hspace{1.5cm}

\begin{questions}
\question[20] PigD

\begin{oneparchoices}
\choice  2
\choice  1
\choice  3
\CorrectChoice 4
\end{oneparchoices}

\end{questions}

\hspace{5cm}

\section{\normalsize{主观题 (本大题共 1 题,满分 30 分)}}
\hspace{1.5cm}

\begin{questions}
\question[30] 赏析“可怜无定河边骨,犹是春闺梦里人”

参考答案: \textbf{此句没有直接战争带来的悲惨景象,也没有渲染家人的悲伤情绪,而是匠心独运,把“河边骨”和“春闺梦”联系起来,写闺中妻子不知征人战死,仍然在梦中想见已成白骨的丈夫,使全诗产生震撼心灵的悲剧力量。}

\end{questions}

\end{document}

