
\documentclass[12pt, a4paper, addpoints]{exam}
\usepackage{xeCJK}
\setCJKmainfont{SimSun}[BoldFont=SimHei,ItalicFont=KaiTi]

\footer{}{第 \thepage 页 (共 \pageref{LastPage} 页)}{}
\firstpageheader{诗句默写 / 2023古诗文}{}{姓名:\qquad\qquad}
\runningheader{}{诗句默写 / 2023古诗文}{}

\usepackage{multicol}
\usepackage{lastpage}

\usepackage[
pdfa=true,
unicode=true,
hidelinks,
pdfauthor={李沪纲},
pdftitle={诗句默写 / 2023古诗文},
pdfsubject={诗句默写 / 2023古诗文}]{hyperref}

\usepackage{geometry}
\geometry{a4paper,scale=0.8}

\usepackage{ulem}

% \usepackage{fancyhdr}
% \pagestyle{fancy}
% \fancyhead[L]{诗句默写 / 2023古诗文}
% \fancyhead[R]{班级:\qquad 姓名:\qquad\qquad}
% \fancyfoot[C]{第 \thepage 页 (共 \pageref{LastPage} 页)}

\pointformat{(\thepoints)}
\pointname{分}

\setlength{\parindent}{2em}

\begin{document}

\pagestyle{headandfoot}

\begin{center}
\fbox{\fbox{\parbox{5.5in}{\centering
诗句默写 / 2023古诗文

命题人:李沪纲

(答题时间 10 分钟,共 50 题,满分 96 分)

于 2023/7/17 17:32:47 导出}}}
\end{center}
\vspace{5mm}

\normalsize
\vspace{5mm}

\section{\normalsize{填空题 (本大题共 50 题,满分 96 分)}}
\hspace{1.5cm}

\begin{questions}
\question[2] 人世几回伤往事,\uline{\qquad\qquad\qquad\qquad}。

\question[2] 饮马渡秋水,\uline{\qquad\qquad\qquad\qquad}。

\question[2] 平沙日未没,\uline{\qquad\qquad\qquad\qquad}。

\question[2] \uline{\qquad\qquad\qquad\qquad},徙倚欲何依。

\question[2] 黄尘足今古,\uline{\qquad\qquad\qquad\qquad}。

\question[2] 长天落霞,\uline{\qquad\qquad\qquad\qquad},老树昏鸦。

\question[2] \uline{\qquad\qquad\qquad\qquad},我辈复登临。

\question[2] \uline{\qquad\qquad\qquad\qquad},连山接海隅。

\question[2] \uline{\qquad\qquad\qquad\qquad},午醉醒来愁未醒。送春春去几时回?

\question[2] 丞相祠堂何处寻?\uline{\qquad\qquad\qquad\qquad}。

\question[2] 山随平野尽,\uline{\qquad\qquad\qquad\qquad}。

\question[2] \uline{\qquad\qquad\qquad\qquad},相逢每醉还。

\question[2] 牧人驱犊返,\uline{\qquad\qquad\qquad\qquad}。

\question[2] 几句杜陵诗,\uline{\qquad\qquad\qquad\qquad}。

\question[2] 水调数声持酒听,午醉醒来愁未醒。\uline{\qquad\qquad\qquad\qquad}?

\question[1] \uline{\qquad\qquad\qquad\qquad},萧疏鬓已斑。

\question[2] \uline{\qquad\qquad\qquad\qquad},更吹落、星如雨。

\question[2] \uline{\qquad\qquad\qquad\qquad},涵虚混太清。

\question[2] 槛菊愁烟兰泣露,\uline{\qquad\qquad\qquad\qquad},燕子双飞去。

\question[2] \uline{\qquad\qquad\qquad\qquad},狂歌五柳前。

\question[2] \uline{\qquad\qquad\qquad\qquad},小舟撑出柳阴来。

\question[2] 郭边万户皆临水,\uline{\qquad\qquad\qquad\qquad}。

\question[2] \uline{\qquad\qquad\qquad\qquad},天寒梦泽深。

\question[2] \uline{\qquad\qquad\qquad\qquad},阴晴众壑殊。

\question[2] 太乙近天都,\uline{\qquad\qquad\qquad\qquad}。

\question[2] 蛾儿雪柳黄金缕,\uline{\qquad\qquad\qquad\qquad}。

\question[2] \uline{\qquad\qquad\qquad\qquad},闲敲棋子落灯花。

\question[2] \uline{\qquad\qquad\qquad\qquad},山形依旧枕寒流。

\question[2] 惊起却回头,\uline{\qquad\qquad\qquad\qquad}。

\question[2] \uline{\qquad\qquad\qquad\qquad},山长水阔知何处?

\question[2] 白鸟一双临水立,\uline{\qquad\qquad\qquad\qquad}。

\question[2] 万壑有声含晚籁,\uline{\qquad\qquad\qquad\qquad}。

\question[2] 宝马雕车香满路。\uline{\qquad\qquad\qquad\qquad},玉壶光转,一夜鱼龙舞。

\question[2] \uline{\qquad\qquad\qquad\qquad},青霭入看无。

\question[2] 诗家清景在新春,\uline{\qquad\qquad\qquad\qquad}。

\question[2] 仍怜故乡水,\uline{\qquad\qquad\qquad\qquad}。

\question[1] 浮云一别后,\uline{\qquad\qquad\qquad\qquad}。

\question[2] \uline{\qquad\qquad\qquad\qquad},村桥原树似吾乡。

\question[2] \uline{\qquad\qquad\qquad\qquad}?夹岸桃花蘸水开。

\question[2] \uline{\qquad\qquad\qquad\qquad},寂寞沙洲冷。

\question[2] 林表明霁色,\uline{\qquad\qquad\qquad\qquad}。

\question[2] \uline{\qquad\qquad\qquad\qquad},徒有羡鱼情。

\question[1] \uline{\qquad\qquad\qquad\qquad},云山况是客中过。

\question[1] 江汉曾为客,\uline{\qquad\qquad\qquad\qquad}。

\question[2] 人事有代谢,\uline{\qquad\qquad\qquad\qquad}。

\question[2] \uline{\qquad\qquad\qquad\qquad},雪后千峰半入城。

\question[2] 重重帘幕密遮灯,风不定,人初静,\uline{\qquad\qquad\qquad\qquad}。

\question[2] 宝马雕车香满路。凤箫声动,玉壶光转,\uline{\qquad\qquad\qquad\qquad}。

\question[2] \uline{\qquad\qquad\qquad\qquad},长歌怀采薇。

\question[2] 何事吟馀忽惆怅,\uline{\qquad\qquad\qquad\qquad}。

\end{questions}

\end{document}

