
\documentclass[12pt, a4paper, addpoints]{exam}
\usepackage{xeCJK}

\usepackage{fancyhdr}
\usepackage{lastpage}

\usepackage[
pdfa=true,
unicode=true,
hidelinks,
pdfauthor={李沪纲},
pdftitle={小测试},
pdfsubject={小测试}]{hyperref}

\usepackage{geometry}
\geometry{a4paper,scale=0.8}

\usepackage{ulem}

\pagestyle{fancy}
\fancyhead[L]{小测试}
\fancyhead[R]{班级:\qquad 姓名:\qquad\qquad}
\fancyfoot[C]{第 \thepage 页 (共 \pageref{LastPage} 页)}

\pointformat{(\thepoints)}
\pointname{分}

\setlength{\parindent}{2em}

\begin{document}
\begin{center}
\fbox{\fbox{\parbox{5.5in}{\centering
小测试

命题人:李沪纲

(答题时间 3 分钟,共 5 题,满分 14 分)

于 2023/7/16 17:33:48 导出}}}
\end{center}
\vspace{5mm}

\normalsize
\vspace{5mm}

\section{单选题 (本大题共 2 题,满分 4 分)}
\hspace{1.5cm}
\begin{questions}
\question[2] 测试

\begin{oneparchoices}
\choice A
\choice B
\choice C
\choice D
\end{oneparchoices}

标准答案: C

\question[2] 你好啊

\begin{oneparchoices}
\choice A
\choice B
\choice C
\choice D
\end{oneparchoices}

标准答案: B

\end{questions}

\hspace{5cm}

\section{多选题 (本大题共 1 题,满分 3 分)}
\hspace{1.5cm}
\begin{questions}
\question[3] 以下说法正确的是

\begin{checkboxes}
\choice 地球是圆的
\choice 中华有上下五千年历史
\choice 史记里没有\textbf{郭子仪}传
\choice 历史考试有\textit{30}分
\end{checkboxes}

标准答案: 地球是圆的,中华有上下五千年历史,史记里没有\textbf{郭子仪}传,历史考试有\textit{}分

\end{questions}

\hspace{5cm}

\section{填空题 (本大题共 1 题,满分 2 分)}
\hspace{1.5cm}
\begin{questions}
\question[2] 可怜无定河边骨,可怜无定河边骨,犹是\uline{\qquad\qquad\qquad\qquad}梦里人
标准答案: 春闺

\end{questions}

\hspace{5cm}

\section{主观题 (本大题共 1 题,满分 5 分)}
\hspace{1.5cm}
\begin{questions}
\question[5] 赏析“气蒸云梦泽,波撼岳阳城”一句
\vspace{\stretch{1}}
参考答案: 好

\end{questions}

\end{document}

