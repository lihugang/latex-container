
\documentclass[12pt, a4paper, addpoints]{exam}
\usepackage{xeCJK}
\setCJKmainfont{SimSun}[BoldFont=SimHei,ItalicFont=KaiTi]

\footer{}{第 \thepage 页 (共 \pageref{LastPage} 页)}{}
\firstpageheader{小测试}{}{姓名:\quad\textbf{pig}}
\runningheader{}{小测试}{}

\usepackage{multicol}
\usepackage{lastpage}

\usepackage[
pdfa=true,
unicode=true,
hidelinks,
pdfauthor={李沪纲},
pdftitle={小测试},
pdfsubject={小测试}]{hyperref}

\usepackage{geometry}
\geometry{a4paper,scale=0.8}

\usepackage{ulem}

\pointformat{(\thepoints)}
\pointname{分}

\setlength{\parindent}{2em}

\begin{document}

\pagestyle{headandfoot}

\begin{center}
\fbox{\fbox{\parbox{5.5in}{\centering
小测试

命题人:李沪纲

(答题时间 5 分钟,共 4 题,满分 70 分)

于 2023/7/21 11:23:35 导出}}}
\end{center}
\vspace{5mm}

\normalsize
\vspace{5mm}

\section{\normalsize{单选题 (本大题共 1 题,满分 20 分)}}
\hspace{1.5cm}

\begin{questions}
\question[20] \textbf{选A}

\begin{oneparchoices}
\choice 1
\choice 2
\choice 3
\choice 4
\end{oneparchoices}
我的答案:1

\end{questions}

\hspace{5cm}

\section{\normalsize{多选题 (本大题共 1 题,满分 10 分)}}
\hspace{1.5cm}
\begin{multicols}{2}
\begin{questions}
\question[10] 全选

\begin{checkboxes}
\choice 1
\choice 2
\choice 3
\choice \textbf{4}
\end{checkboxes}

我的答案:\textbf{4},1,2,3

\end{questions}
\end{multicols}

\hspace{5cm}

\section{\normalsize{填空题 (本大题共 1 题,满分 20 分)}}
\hspace{1.5cm}

\begin{questions}
\question[20] 可怜\uline{\qquad\qquad\qquad}边骨,犹是春闺梦里人

我的答案:无定河

\end{questions}

\hspace{5cm}

\section{\normalsize{主观题 (本大题共 1 题,满分 20 分)}}
\hspace{1.5cm}

\begin{questions}
\question[20] 赏析“可怜无定河边骨,犹是春闺梦里人”
https://gsw.lihugang.top

\vspace{\stretch{1}}

我的答案:不好

\end{questions}

\end{document}

