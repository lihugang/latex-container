
\documentclass[12pt, a4paper, addpoints]{exam}
\usepackage{xeCJK}
\setCJKmainfont{SimSun}[BoldFont=SimHei,ItalicFont=KaiTi]

\footer{}{第 \thepage 页 (共 \pageref{LastPage} 页)}{}
\firstpageheader{诗句默写 / 2023古诗文}{}{姓名:\qquad\qquad}
\runningheader{}{诗句默写 / 2023古诗文}{}

\usepackage{lastpage}

\usepackage[
pdfa=true,
unicode=true,
hidelinks,
pdfauthor={李沪纲},
pdftitle={诗句默写 / 2023古诗文},
pdfsubject={诗句默写 / 2023古诗文}]{hyperref}

\usepackage{geometry}
\geometry{a4paper,scale=0.8}

\usepackage{ulem}

% \usepackage{fancyhdr}
% \pagestyle{fancy}
% \fancyhead[L]{诗句默写 / 2023古诗文}
% \fancyhead[R]{班级:\qquad 姓名:\qquad\qquad}
% \fancyfoot[C]{第 \thepage 页 (共 \pageref{LastPage} 页)}

\pointformat{(\thepoints)}
\pointname{分}

\setlength{\parindent}{2em}

\begin{document}

\pagestyle{headandfoot}

\begin{center}
\fbox{\fbox{\parbox{5.5in}{\centering
诗句默写 / 2023古诗文

命题人:李沪纲

(答题时间 10 分钟,共 50 题,满分 100 分)

于 2023/7/17 16:48:46 导出}}}
\end{center}
\vspace{5mm}

\normalsize
\vspace{5mm}

\section{\normalsize{填空题 (本大题共 50 题,满分 100 分)}}
\hspace{1.5cm}
\begin{questions}
\question[2] \uline{\qquad\qquad\qquad\qquad},我辈复登临。

\question[2] 棠梨叶落胭脂色,\uline{\qquad\qquad\qquad\qquad}。

\question[2] \uline{\qquad\qquad\qquad\qquad},来从楚国游。

\question[2] 宝马雕车香满路。\uline{\qquad\qquad\qquad\qquad},玉壶光转,一夜鱼龙舞。

\question[2] \uline{\qquad\qquad\qquad\qquad},见人惊起入芦花。

\question[2] 水落鱼梁浅,\uline{\qquad\qquad\qquad\qquad}。

\question[2] 有约不来过夜半,\uline{\qquad\qquad\qquad\qquad}。

\question[2] 若待上林花似锦,\uline{\qquad\qquad\qquad\qquad}。

\question[2] 朱楼四面钩疏箔,\uline{\qquad\qquad\qquad\qquad}。

\question[2] 谁见幽人独往来,\uline{\qquad\qquad\qquad\qquad}。

\question[2] 牧人驱犊返,\uline{\qquad\qquad\qquad\qquad}。

\question[2] \uline{\qquad\qquad\qquad\qquad},狂歌五柳前。

\question[2] \uline{\qquad\qquad\qquad\qquad}?锦官城外柏森森。

\question[2] \uline{\qquad\qquad\qquad\qquad},北风吹起数声雷。

\question[2] 倚杖柴门外,\uline{\qquad\qquad\qquad\qquad}。

\question[2] 拣尽寒枝不肯栖,\uline{\qquad\qquad\qquad\qquad}。

\question[2] 分野中峰变,\uline{\qquad\qquad\qquad\qquad}。

\question[2] \uline{\qquad\qquad\qquad\qquad},独上高楼,望尽天涯路。

\question[2] \uline{\qquad\qquad\qquad\qquad},水寒风似刀。

\question[2] \uline{\qquad\qquad\qquad\qquad},天寒梦泽深。

\question[2] 众里寻他千百度,蓦然回首,\uline{\qquad\qquad\qquad\qquad},灯火阑珊处。

\question[2] 众里寻他千百度,\uline{\qquad\qquad\qquad\qquad},那人却在,灯火阑珊处。

\question[2] 闺中少妇不知愁,\uline{\qquad\qquad\qquad\qquad}。

\question[2] 缺月挂疏桐,\uline{\qquad\qquad\qquad\qquad}。

\question[2] \uline{\qquad\qquad\qquad\qquad},云生结海楼。

\question[2] \uline{\qquad\qquad\qquad\qquad},出门俱是看花人。

\question[2] \uline{\qquad\qquad\qquad\qquad},江入大荒流。

\question[2] 相顾无相识,\uline{\qquad\qquad\qquad\qquad}。

\question[2] 王濬楼船下益州,\uline{\qquad\qquad\qquad\qquad}。

\question[2] \uline{\qquad\qquad\qquad\qquad},春日凝妆上翠楼。

\question[2] \uline{\qquad\qquad\qquad\qquad},猎马带禽归。

\question[2] \uline{\qquad\qquad\qquad\qquad},潮退渔船阁岸斜。

\question[2] 复值接舆醉,\uline{\qquad\qquad\qquad\qquad}。

\question[2] 树树皆秋色,\uline{\qquad\qquad\qquad\qquad}。

\question[2] 三顾频烦天下计,\uline{\qquad\qquad\qquad\qquad}。

\question[2] 太乙近天都,\uline{\qquad\qquad\qquad\qquad}。

\question[2] 东皋薄暮望,\uline{\qquad\qquad\qquad\qquad}。

\question[2] 昔日长城战,\uline{\qquad\qquad\qquad\qquad}。

\question[2] \uline{\qquad\qquad\qquad\qquad},罗幕轻寒,燕子双飞去。

\question[2] 山郡逢春复乍晴,\uline{\qquad\qquad\qquad\qquad}?

\question[2] 东风夜放花千树,\uline{\qquad\qquad\qquad\qquad}。

\question[2] 白鸟一双临水立,\uline{\qquad\qquad\qquad\qquad}。

\question[2] 惊起却回头,\uline{\qquad\qquad\qquad\qquad}。

\question[2] 欲寄彩笺兼尺素,\uline{\qquad\qquad\qquad\qquad}?

\question[2] 宝马雕车香满路。凤箫声动,\uline{\qquad\qquad\qquad\qquad},一夜鱼龙舞。

\question[2] \uline{\qquad\qquad\qquad\qquad},信马悠悠野兴长。

\question[2] 丞相祠堂何处寻?\uline{\qquad\qquad\qquad\qquad}。

\question[2] 诗家清景在新春,\uline{\qquad\qquad\qquad\qquad}。

\question[2] 欲济无舟楫,\uline{\qquad\qquad\qquad\qquad}。

\question[2] 郭边万户皆临水,\uline{\qquad\qquad\qquad\qquad}。

\end{questions}

\end{document}

