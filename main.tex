
\documentclass[12pt, a4paper, addpoints, answers]{exam}
\printanswers

\usepackage{xeCJK}
\setCJKmainfont{SimSun}[BoldFont=SimHei,ItalicFont=KaiTi]

\footer{}{第 \thepage 页 (共 \pageref{LastPage} 页)}{}
\firstpageheader{Test}{}{姓名:\quad\textbf{李沪纲}}
\runningheader{}{Test}{}

\usepackage{amssymb}
\usepackage{multicol}
\usepackage{lastpage}

\usepackage{xcolor}

\usepackage[
pdfa=true,
unicode=true,
hidelinks,
pdfauthor={李沪纲},
pdftitle={Test},
pdfsubject={Test}]{hyperref}

\usepackage{geometry}
\geometry{a4paper,scale=0.8}

\usepackage{ulem}

\pointformat{(\thepoints)}
\pointname{分}
\checkboxchar{$\square$}
\checkedchar{$\blacksquare$}

\setlength{\parindent}{2em}

\begin{document}

\pagestyle{headandfoot}

\begin{center}
\fbox{\fbox{\parbox{5.5in}{\centering
Test

命题人:李沪纲

(答题时间 2 分钟,共 \numquestions 题,满分 \numpoints 分)

于 2023/7/25 14:34:31 导出}}}
\end{center}
\vspace{5mm}

\normalsize
\vspace{5mm}
得分:判题未完成\quad 答题用时:2分07秒
\hspace{2cm}

\section{\normalsize{填空题 (本大题共 20 题,满分 60 分)}}
\hspace{1.5cm}

\begin{questions}
\question[3] 年年战骨埋荒外,\fillin 。

   

 

\question[3] 何因不归去?\fillin 。

   

 

\question[3] 一日一日似流水,\fillin 。

   

 

\question[3] 临晚镜,伤流景,\fillin 。

   

 

\question[3] 山下兰芽短浸溪,\fillin ,萧萧暮雨子规啼。

   

 

\question[3] 长天落霞,\fillin ,老树昏鸦。

   

 

\question[3] 寒风夕吹易水波,\fillin 。

   

 

\question[3] 胡雁哀鸣夜夜飞,\fillin 。

   

 

\question[3] \fillin ,相逢每醉还。

   

 

\question[3] \fillin ,笑语盈盈暗香去。

   

 

\question[3] 年年战骨埋荒外,\fillin 。

   

 

\question[3] 岂有千山与万山,\fillin 。

   

 

\question[3] \fillin ,何如此处学长生。

   

 

\question[3] \fillin ,无计啭春风。

   

 

\question[3] 马穿山径菊初黄,\fillin 。

   

 

\question[3] \fillin ,一片降幡出石头。

   

 

\question[3] 海浪如云去却回,\fillin[北风吹起数声雷] 。

   

 

\question[3] \fillin[春水断桥人不度] ,小舟撑出柳阴来。

   

 

\question[3] 长天落霞,\fillin ,老树昏鸦。

   

 

\question[3] 岂有千山与万山,\fillin 。

   

 
\end{questions}

\hspace{5cm}

\section{\normalsize{填空题 (本大题共 10 题,满分 20 分)}}
\hspace{1.5cm}
\begin{multicols}{2}
\begin{questions}
\question[2] 蜗牛角上\fillin[争何事] 

   

 

\question[2] \fillin[进可以攻] ,退可以守

   

 

\question[2] \fillin[劳于读书] ,逸于作文

   

 

\question[2] \fillin[君子之交] 淡如水

   

 

\question[2] 同声相应,\fillin[同气相求] 

   

 

\question[2] 天行健,\fillin[君子以自强不息] 

   

 

\question[2] \fillin[江山易改] ,禀性难移

   

 

\question[2] 万人丛中一握手,\fillin[使我衣袖三年香] 

   

 

\question[2] 屋漏\fillin[偏逢] 连夜雨

   

 

\question[2] \fillin 如珠

   

 
\end{questions}
\end{multicols}

\hspace{5cm}

\section{\normalsize{多选题 (本大题共 4 题,满分 20 分)}}
\hspace{1.5cm}

\begin{questions}
\question[5] C

\begin{oneparcheckboxes}
\choice  1
\choice  2
\CorrectChoice 3
\choice  4
\end{oneparcheckboxes}

\question[5] C

\begin{oneparcheckboxes}
\choice  1
\choice  2
\CorrectChoice 3
\choice  4
\end{oneparcheckboxes}

\question[5] AB

\begin{oneparcheckboxes}
\CorrectChoice 1
\CorrectChoice 2
\choice  3
\choice  4
\end{oneparcheckboxes}

\question[5] AB

\begin{oneparcheckboxes}
\CorrectChoice 1
\CorrectChoice 2
\choice  3
\choice  4
\end{oneparcheckboxes}

\end{questions}

\hspace{5cm}

\section{\normalsize{单选题 (本大题共 1 题,满分 20 分)}}
\hspace{1.5cm}

\begin{questions}
\question[20] PigD

\begin{oneparchoices}
\choice  2
\choice  1
\choice  3
\CorrectChoice 4
\end{oneparchoices}

\end{questions}

\hspace{5cm}

\section{\normalsize{主观题 (本大题共 1 题,满分 30 分)}}
\hspace{1.5cm}

\begin{questions}
\question[30] 赏析“可怜无定河边骨,犹是春闺梦里人”

\fillwithlines{\stretch{1}}

得分:等待判卷   

\end{questions}

\end{document}

