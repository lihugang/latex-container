
\documentclass[12pt, a4paper, addpoints]{exam}
\usepackage{xeCJK}

\usepackage{lastpage}

\usepackage[
pdfa=true,
unicode=true,
hidelinks,
pdfauthor={李沪纲},
pdftitle={诗句默写 / 2023古诗文},
pdfsubject={诗句默写 / 2023古诗文}]{hyperref}

\usepackage{geometry}
\geometry{a4paper,scale=0.8}

\usepackage{ulem}

% \usepackage{fancyhdr}
% \pagestyle{fancy}
% \fancyhead[L]{诗句默写 / 2023古诗文}
% \fancyhead[R]{班级:\qquad 姓名:\qquad\qquad}
% \fancyfoot[C]{第 \thepage 页 (共 \pageref{LastPage} 页)}

\pointformat{(\thepoints)}
\pointname{分}

\setlength{\parindent}{2em}

\begin{document}

\pagestyle{headandfoot}
\runningheadrule
\firstpageheader{诗句默写 / 2023古诗文}{}{姓名:\qquad\qquad}
\runningheader{诗句默写 / 2023古诗文}
\firstpagefooter{}{第 \thepage 页 (共 \pageref{LastPage} 页)}{}
\runningfooter{}{第 \thepage 页 (共 \pageref{LastPage} 页)}{}

\begin{center}
\fbox{\fbox{\parbox{5.5in}{\centering
诗句默写 / 2023古诗文

命题人:李沪纲

(答题时间 10 分钟,共 50 题,满分 100 分)

于 2023/7/16 22:48:08 导出}}}
\end{center}
\vspace{5mm}

\normalsize
\vspace{5mm}

\section{\normalsize{填空题 (本大题共 50 题,满分 100 分)}}
\hspace{1.5cm}
\begin{questions}
\question[2] \uline{\qquad\qquad\qquad\qquad},独上高楼,望尽天涯路。

\question[2] \uline{\qquad\qquad\qquad\qquad},金陵王气黯然收。

\question[2] \uline{\qquad\qquad\qquad\qquad},江入大荒流。

\question[2] 欲寄彩笺兼尺素,\uline{\qquad\qquad\qquad\qquad}?

\question[2] \uline{\qquad\qquad\qquad\qquad},阴晴众壑殊。

\question[2] \uline{\qquad\qquad\qquad\qquad},闲敲棋子落灯花。

\question[2] \uline{\qquad\qquad\qquad\qquad},春日凝妆上翠楼。

\question[2] 谁见幽人独往来,\uline{\qquad\qquad\qquad\qquad}。

\question[2] 双飞燕子几时回?\uline{\qquad\qquad\qquad\qquad}。

\question[2] \uline{\qquad\qquad\qquad\qquad},山长水阔知何处?

\question[2] \uline{\qquad\qquad\qquad\qquad},积雪浮云端。

\question[2] 若待上林花似锦,\uline{\qquad\qquad\qquad\qquad}。

\question[2] 东皋薄暮望,\uline{\qquad\qquad\qquad\qquad}。

\question[2] \uline{\qquad\qquad\qquad\qquad},山形依旧枕寒流。

\question[2] 槛菊愁烟兰泣露,罗幕轻寒,\uline{\qquad\qquad\qquad\qquad}。

\question[2] \uline{\qquad\qquad\qquad\qquad},斜光到晓穿朱户。

\question[2] \uline{\qquad\qquad\qquad\qquad},出门俱是看花人。

\question[2] 丞相祠堂何处寻?\uline{\qquad\qquad\qquad\qquad}。

\question[2] 林表明霁色,\uline{\qquad\qquad\qquad\qquad}。

\question[2] 昨夜西风凋碧树,\uline{\qquad\qquad\qquad\qquad},望尽天涯路。

\question[2] 东风夜放花千树,\uline{\qquad\qquad\qquad\qquad}。

\question[2] 渡远荆门外,\uline{\qquad\qquad\qquad\qquad}。

\question[2] \uline{\qquad\qquad\qquad\qquad},秋水日潺湲。

\question[2] 黄梅时节家家雨,\uline{\qquad\qquad\qquad\qquad}。

\question[2] 拣尽寒枝不肯栖,\uline{\qquad\qquad\qquad\qquad}。

\question[2] 有约不来过夜半,\uline{\qquad\qquad\qquad\qquad}。

\question[2] \uline{\qquad\qquad\qquad\qquad},万里送行舟。

\question[2] \uline{\qquad\qquad\qquad\qquad},读罢泪沾襟。

\question[2] 渡头馀落日,\uline{\qquad\qquad\qquad\qquad}。

\question[2] \uline{\qquad\qquad\qquad\qquad},悔教夫婿觅封侯。

\question[2] 八月湖水平,\uline{\qquad\qquad\qquad\qquad}。

\question[2] \uline{\qquad\qquad\qquad\qquad},黯黯见临洮。

\question[2] 万壑有声含晚籁,\uline{\qquad\qquad\qquad\qquad}。

\question[2] \uline{\qquad\qquad\qquad\qquad},故垒萧萧芦荻秋。

\question[2] 宝马雕车香满路。\uline{\qquad\qquad\qquad\qquad},玉壶光转,一夜鱼龙舞。

\question[2] \uline{\qquad\qquad\qquad\qquad}?锦官城外柏森森。

\question[2] 山随平野尽,\uline{\qquad\qquad\qquad\qquad}。

\question[2] 复值接舆醉,\uline{\qquad\qquad\qquad\qquad}。

\question[2] 山郡逢春复乍晴,\uline{\qquad\qquad\qquad\qquad}?

\question[2] 江山留胜迹,\uline{\qquad\qquad\qquad\qquad}。

\question[2] \uline{\qquad\qquad\qquad\qquad},数峰无语立斜阳。

\question[2] 马穿山径菊初黄,\uline{\qquad\qquad\qquad\qquad}。

\question[2] 树树皆秋色,\uline{\qquad\qquad\qquad\qquad}。

\question[2] 宝马雕车香满路。凤箫声动,\uline{\qquad\qquad\qquad\qquad},一夜鱼龙舞。

\question[2] \uline{\qquad\qquad\qquad\qquad},长使英雄泪满襟。

\question[2] 相顾无相识,\uline{\qquad\qquad\qquad\qquad}。

\question[2] \uline{\qquad\qquad\qquad\qquad},往来成古今。

\question[2] \uline{\qquad\qquad\qquad\qquad},天寒梦泽深。

\question[2] \uline{\qquad\qquad\qquad\qquad},临风听暮蝉。

\question[2] 欲投人处宿,\uline{\qquad\qquad\qquad\qquad}。

\end{questions}

\end{document}

