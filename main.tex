
\documentclass[12pt, a4paper, addpoints, answers]{exam}
\printanswers

\usepackage{xeCJK}
\setCJKmainfont{SimSun}[BoldFont=SimHei,ItalicFont=KaiTi]

\footer{}{第 \thepage 页 (共 \pageref{LastPage} 页)}{}
\firstpageheader{Test}{}{姓名:\quad\textbf{李沪纲}}
\runningheader{}{Test}{}

\usepackage{amssymb}
\usepackage{multicol}
\usepackage{lastpage}

\usepackage{xcolor}

\usepackage[
pdfa=true,
unicode=true,
hidelinks,
pdfauthor={李沪纲},
pdftitle={Test},
pdfsubject={Test}]{hyperref}

\usepackage{geometry}
\geometry{a4paper,scale=0.8}

\usepackage{ulem}

\pointformat{(\thepoints)}
\pointname{分}
\checkboxchar{$\square$}
\checkedchar{$\blacksquare$}

\setlength{\parindent}{2em}

\begin{document}

\pagestyle{headandfoot}

\begin{center}
\fbox{\fbox{\parbox{5.5in}{\centering
Test

命题人:李沪纲

(答题时间 2 分钟,共 \numquestions 题,满分 \numpoints 分)

于 2023/7/25 14:26:47 导出}}}
\end{center}
\vspace{5mm}

\normalsize
\vspace{5mm}
得分:判题未完成\quad 答题用时:0分00秒
\hspace{2cm}

\section{\normalsize{填空题 (本大题共 20 题,满分 60 分)}}
\hspace{1.5cm}

\begin{questions}
\question[3] 分野中峰变,\fillin 。

   

 

\question[3] 树树皆秋色,\fillin 。

   

 

\question[3] 谁道人生无再少?门前流水尚能西!\fillin 。

   

 

\question[3] 众里寻他千百度,蓦然回首,\fillin ,灯火阑珊处。

   

 

\question[3] 寄语东阳沽酒市,拼一醉,\fillin 。

   

 

\question[3] 黄梅时节家家雨,\fillin 。

   

 

\question[3] \fillin ,云破月来花弄影。

   

 

\question[3] \fillin ,御苑砧声向晚多。

   

 

\question[3] \fillin ,枫叶落纷纷。

   

 

\question[3] 诗家清景在新春,\fillin 。

   

 

\question[3] 临晚镜,伤流景,\fillin 。

   

 

\question[3] 山下兰芽短浸溪,松间沙路净无泥,\fillin 。

   

 

\question[3] 气蒸云梦泽,\fillin 。

   

 

\question[3] \fillin ,一片降幡出石头。

   

 

\question[3] 东皋薄暮望,\fillin 。

   

 

\question[3] 借问路旁名利客,\fillin 。

   

 

\question[3] 平沙日未没,\fillin 。

   

 

\question[3] \fillin ,渔郎奠竹杯。

   

 

\question[3] 一日一日似流水,\fillin 。

   

 

\question[3] 何因不归去?\fillin 。

   

 
\end{questions}

\hspace{5cm}

\section{\normalsize{填空题 (本大题共 10 题,满分 20 分)}}
\hspace{1.5cm}
\begin{multicols}{2}
\begin{questions}
\question[2] 地势坤,\fillin 

   

 

\question[2] \fillin ,如见其人

   

 

\question[2] \fillin ,不患无土

   

 

\question[2] 前事不忘,\fillin 

   

 

\question[2] 富贵必从\fillin 

   

 

\question[2] \fillin ,可以明得失

   

 

\question[2] \fillin ,自有留人处

   

 

\question[2] 一语天然\fillin 

   

 

\question[2] \fillin ,俯不怍于人

   

 

\question[2] 毁誉不干其守,\fillin 

   

 
\end{questions}
\end{multicols}

\hspace{5cm}

\section{\normalsize{多选题 (本大题共 4 题,满分 20 分)}}
\hspace{1.5cm}

\begin{questions}
\question[5] AB

\begin{oneparcheckboxes}
\choice  1
\choice  2
\choice  3
\choice  4
\end{oneparcheckboxes}

\question[5] ABCD

\begin{oneparcheckboxes}
\choice  1
\choice  2
\choice  3
\choice  4
\end{oneparcheckboxes}

\question[5] ABCD

\begin{oneparcheckboxes}
\choice  1
\choice  2
\choice  3
\choice  4
\end{oneparcheckboxes}

\question[5] AB

\begin{oneparcheckboxes}
\choice  1
\choice  2
\choice  3
\choice  4
\end{oneparcheckboxes}

\end{questions}

\hspace{5cm}

\section{\normalsize{单选题 (本大题共 1 题,满分 20 分)}}
\hspace{1.5cm}

\begin{questions}
\question[20] PigD

\begin{oneparchoices}
\choice  2
\choice  1
\choice  3
\choice  4
\end{oneparchoices}

\answerline

\end{questions}

\hspace{5cm}

\section{\normalsize{主观题 (本大题共 1 题,满分 30 分)}}
\hspace{1.5cm}

\begin{questions}
\question[30] 赏析“可怜无定河边骨,犹是春闺梦里人”

\fillwithlines{\stretch{1}}

   

\end{questions}

\end{document}

