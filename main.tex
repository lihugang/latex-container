
\documentclass[12pt, a4paper, addpoints]{exam}
\usepackage{xeCJK}
\setCJKmainfont{SimSun}[BoldFont=SimHei,ItalicFont=KaiTi]

\footer{}{第 \thepage 页 (共 \pageref{LastPage} 页)}{}
\firstpageheader{诗句默写 / 2023古诗文}{}{姓名:\qquad\qquad}
\runningheader{}{诗句默写 / 2023古诗文}{}

\usepackage{multicol}
\usepackage{lastpage}

\usepackage[
pdfa=true,
unicode=true,
hidelinks,
pdfauthor={李沪纲},
pdftitle={诗句默写 / 2023古诗文},
pdfsubject={诗句默写 / 2023古诗文}]{hyperref}

\usepackage{geometry}
\geometry{a4paper,scale=0.8}

\usepackage{ulem}

% \usepackage{fancyhdr}
% \pagestyle{fancy}
% \fancyhead[L]{诗句默写 / 2023古诗文}
% \fancyhead[R]{班级:\qquad 姓名:\qquad\qquad}
% \fancyfoot[C]{第 \thepage 页 (共 \pageref{LastPage} 页)}

\pointformat{(\thepoints)}
\pointname{分}

\setlength{\parindent}{2em}

\begin{document}

\pagestyle{headandfoot}

\begin{center}
\fbox{\fbox{\parbox{5.5in}{\centering
诗句默写 / 2023古诗文

命题人:李沪纲

(答题时间 10 分钟,共 50 题,满分 96 分)

于 2023/7/17 18:26:54 导出}}}
\end{center}
\vspace{5mm}

\normalsize
\vspace{5mm}

\section{\normalsize{填空题 (本大题共 50 题,满分 96 分)}}
\hspace{1.5cm}
\begin{multicols}{2}
\begin{questions}
\question[2] 气蒸云梦泽,\uline{\qquad\qquad\qquad}。

\question[2] \uline{\qquad\qquad\qquad},卧看千山急雨来。

\question[2] 映阶碧草自春色,\uline{\qquad\qquad\qquad}。

\question[2] \uline{\qquad\qquad\qquad},信马悠悠野兴长。

\question[2] 谁见幽人独往来,\uline{\qquad\qquad\qquad}。

\question[2] \uline{\qquad\qquad\qquad},临风听暮蝉。

\question[2] \uline{\qquad\qquad\qquad},白骨乱蓬蒿。

\question[2] \uline{\qquad\qquad\qquad},云生结海楼。

\question[2] 槛菊愁烟兰泣露,罗幕轻寒,\uline{\qquad\qquad\qquad}。

\question[1] 何因不归去?\uline{\qquad\qquad\qquad}。

\question[2] \uline{\qquad\qquad\qquad}?锦官城外柏森森。

\question[2] 有约不来过夜半,\uline{\qquad\qquad\qquad}。

\question[2] \uline{\qquad\qquad\qquad},小舟撑出柳阴来。

\question[2] \uline{\qquad\qquad\qquad},金陵王气黯然收。

\question[2] \uline{\qquad\qquad\qquad},客来旋把朱帘挂。

\question[2] 人世几回伤往事,\uline{\qquad\qquad\qquad}。

\question[2] 白鸟一双临水立,\uline{\qquad\qquad\qquad}。

\question[2] \uline{\qquad\qquad\qquad},村桥原树似吾乡。

\question[2] 临晚镜,伤流景,\uline{\qquad\qquad\qquad}。

\question[2] 黄梅时节家家雨,\uline{\qquad\qquad\qquad}。

\question[2] \uline{\qquad\qquad\qquad},春日凝妆上翠楼。

\question[2] 今逢四海为家日,\uline{\qquad\qquad\qquad}。

\question[2] \uline{\qquad\qquad\qquad},城中增暮寒。

\question[2] \uline{\qquad\qquad\qquad},风不定,人初静,明日落红应满径。

\question[2] 白云回望合,\uline{\qquad\qquad\qquad}。

\question[2] \uline{\qquad\qquad\qquad},青草池塘处处蛙。

\question[2] \uline{\qquad\qquad\qquad}。凤箫声动,玉壶光转,一夜鱼龙舞。

\question[2] \uline{\qquad\qquad\qquad},黯黯见临洮。

\question[2] \uline{\qquad\qquad\qquad},长使英雄泪满襟。

\question[1] \uline{\qquad\qquad\qquad}?淮上有秋山。

\question[2] \uline{\qquad\qquad\qquad}?夹岸桃花蘸水开。

\question[2] \uline{\qquad\qquad\qquad},-幅王维画。

\question[2] 复值接舆醉,\uline{\qquad\qquad\qquad}。

\question[2] \uline{\qquad\qquad\qquad},读罢泪沾襟。

\question[2] 人事有代谢,\uline{\qquad\qquad\qquad}。

\question[2] \uline{\qquad\qquad\qquad},徒有羡鱼情。

\question[2] 倚杖柴门外,\uline{\qquad\qquad\qquad}。

\question[2] \uline{\qquad\qquad\qquad},山长水阔知何处?

\question[2] 拣尽寒枝不肯栖,\uline{\qquad\qquad\qquad}。

\question[2] 闺中少妇不知愁,\uline{\qquad\qquad\qquad}。

\question[2] \uline{\qquad\qquad\qquad},往事后期空记省。

\question[1] 莫见长安行乐处,\uline{\qquad\qquad\qquad}。

\question[1] \uline{\qquad\qquad\qquad},云山况是客中过。

\question[2] 树树皆秋色,\uline{\qquad\qquad\qquad}。

\question[2] \uline{\qquad\qquad\qquad},狂歌五柳前。

\question[2] \uline{\qquad\qquad\qquad},两朝开济老臣心。

\question[2] 众里寻他千百度,蓦然回首,\uline{\qquad\qquad\qquad},灯火阑珊处。

\question[2] 牧人驱犊返,\uline{\qquad\qquad\qquad}。

\question[2] \uline{\qquad\qquad\qquad},往来成古今。

\question[2] 水落鱼梁浅,\uline{\qquad\qquad\qquad}。

\end{questions}
\end{multicols}

\end{document}

