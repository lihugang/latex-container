
\documentclass[12pt, a4paper, addpoints, answers]{exam}
\printanswers

\usepackage{xeCJK}
\setCJKmainfont{SimSun}[BoldFont=SimHei,ItalicFont=KaiTi]

\footer{}{第 \thepage 页 (共 \pageref{LastPage} 页)}{}
\firstpageheader{Test}{}{\textbf{标准答案}}
\runningheader{}{Test}{\textbf{标准答案}}

\usepackage{amssymb}
\usepackage{multicol}
\usepackage{lastpage}

\usepackage{xcolor}

\usepackage[
pdfa=true,
unicode=true,
hidelinks,
pdfauthor={李沪纲},
pdftitle={Test},
pdfsubject={Test}]{hyperref}

\usepackage{geometry}
\geometry{a4paper,scale=0.8}

\usepackage{ulem}

\pointformat{(\thepoints)}
\pointname{分}
\checkboxchar{$\square$}
\checkedchar{$\blacksquare$}

\setlength{\parindent}{2em}

\begin{document}

\pagestyle{headandfoot}

\begin{center}
\fbox{\fbox{\parbox{5.5in}{\centering
Test

\textbf{标准答案}

命题人:李沪纲

(答题时间 2 分钟,共 \numquestions 题,满分 \numpoints 分)

于 2023/7/25 12:50:41 导出}}}
\end{center}
\vspace{5mm}

\normalsize
\vspace{5mm}

\section{\normalsize{填空题 (本大题共 299 题,满分 897 分)}}
\hspace{1.5cm}

\begin{questions}
\question[3] \fillin诗家清景在新春,绿柳才黄半未匀。

\question[3] 诗家清景在新春,\fillin绿柳才黄半未匀。

\question[3] \fillin若待上林花似锦,出门俱是看花人。

\question[3] 若待上林花似锦,\fillin出门俱是看花人。

\question[3] \fillin闺中少妇不知愁,春日凝妆上翠楼。

\question[3] 闺中少妇不知愁,\fillin春日凝妆上翠楼。

\question[3] \fillin忽见陌头杨柳色,悔教夫婿觅封侯。

\question[3] 忽见陌头杨柳色,\fillin悔教夫婿觅封侯。

\question[3] \fillin终南阴岭秀,积雪浮云端。

\question[3] 终南阴岭秀,\fillin积雪浮云端。

\question[3] \fillin林表明霁色,城中增暮寒。

\question[3] 林表明霁色,\fillin城中增暮寒。

\question[3] \fillin双飞燕子几时回?夹岸桃花蘸水开。

\question[3] 双飞燕子几时回?\fillin夹岸桃花蘸水开。

\question[3] \fillin春雨断桥人不度,小舟撑出柳阴来。

\question[3] 春雨断桥人不度,\fillin小舟撑出柳阴来。

\question[3] \fillin黄梅时节家家雨,青草池塘处处蛙。

\question[3] 黄梅时节家家雨,\fillin青草池塘处处蛙。

\question[3] \fillin有约不来过夜半,闲敲棋子落灯花。

\question[3] 有约不来过夜半,\fillin闲敲棋子落灯花。

\question[3] \fillin江头落日照平沙,潮退渔船阁岸斜。

\question[3] 江头落日照平沙,\fillin潮退渔船阁岸斜。

\question[3] \fillin白鸟一双临水立,见人惊起入芦花。

\question[3] 白鸟一双临水立,\fillin见人惊起入芦花。

\question[3] \fillin海浪如云去却回,北风吹起数声雷。

\question[3] 海浪如云去却回,\fillin北风吹起数声雷。

\question[3] \fillin朱楼四面钩疏箔,卧看千山急雨来。

\question[3] 朱楼四面钩疏箔,\fillin卧看千山急雨来。

\question[3] \fillin山郡逢春复乍晴,陂塘分出几泉清?

\question[3] 山郡逢春复乍晴,\fillin陂塘分出几泉清?

\question[3] \fillin郭边万户皆临水,雪后千峰半入城。

\question[3] 郭边万户皆临水,\fillin雪后千峰半入城。

\question[3] \fillin渡远荆门外,来从楚国游。

\question[3] 渡远荆门外,\fillin来从楚国游。

\question[3] \fillin山随平野尽,江入大荒流。

\question[3] 山随平野尽,\fillin江入大荒流。

\question[3] \fillin月下飞天镜,云生结海楼。

\question[3] 月下飞天镜,\fillin云生结海楼。

\question[3] \fillin仍怜故乡水,万里送行舟。

\question[3] 仍怜故乡水,\fillin万里送行舟。

\question[3] \fillin寒山转苍翠,秋水日潺湲。

\question[3] 寒山转苍翠,\fillin秋水日潺湲。

\question[3] \fillin倚杖柴门外,临风听暮蝉。

\question[3] 倚杖柴门外,\fillin临风听暮蝉。

\question[3] \fillin渡头余落日,墟里上孤烟。

\question[3] 渡头馀落日,\fillin墟里上孤烟。

\question[3] \fillin复值接舆醉,狂歌五柳前。

\question[3] 复值接舆醉,\fillin狂歌五柳前。

\question[3] \fillin八月湖水平,涵虚混太清。

\question[3] 八月湖水平,\fillin涵虚混太清。

\question[3] \fillin气蒸云梦泽,波撼岳阳城。

\question[3] 气蒸云梦泽,\fillin波撼岳阳城。

\question[3] \fillin欲济无舟楫,端居耻圣明。

\question[3] 欲济无舟楫,\fillin端居耻圣明。

\question[3] \fillin坐观垂钓者,徒有羡鱼情。

\question[3] 坐观垂钓者,\fillin徒有羡鱼情。

\question[3] \fillin太乙近天都,连山接海隅。

\question[3] 太乙近天都,\fillin连山接海隅。

\question[3] \fillin白云回望合,青霭入看无。

\question[3] 白云回望合,\fillin青霭入看无。

\question[3] \fillin分野中峰变,阴晴众壑殊。

\question[3] 分野中峰变,\fillin阴晴众壑殊。

\question[3] \fillin欲投人处宿,隔水问樵夫。

\question[3] 欲投人处宿,\fillin隔水问樵夫。

\question[3] \fillin丞相祠堂何处寻?锦官城外柏森森。

\question[3] 丞相祠堂何处寻?\fillin锦官城外柏森森。

\question[3] \fillin映阶碧草自春色,隔叶黄鹂空好音。

\question[3] 映阶碧草自春色,\fillin隔叶黄鹂空好音。

\question[3] \fillin三顾频烦天下计,两朝开济老臣心。

\question[3] 三顾频烦天下计,\fillin两朝开济老臣心。

\question[3] \fillin出师未捷身先死,长使英雄泪满襟。

\question[3] 出师未捷身先死,\fillin长使英雄泪满襟。

\question[3] \fillin人事有代谢,往来成古今。

\question[3] 人事有代谢,\fillin往来成古今。

\question[3] \fillin江山留胜迹,我辈复登临。

\question[3] 江山留胜迹,\fillin我辈复登临。

\question[3] \fillin水落鱼梁浅,天寒梦泽深。

\question[3] 水落鱼梁浅,\fillin天寒梦泽深。

\question[3] \fillin羊公碑尚在,读罢泪沾襟。

\question[3] 羊公碑尚在,\fillin读罢泪沾襟。

\question[3] \fillin东皋薄暮望,徙倚欲何依。

\question[3] 东皋薄暮望,\fillin徙倚欲何依。

\question[3] \fillin树树皆秋色,山山唯落晖。

\question[3] 树树皆秋色,\fillin山山唯落晖。

\question[3] \fillin牧人驱犊返,猎马带禽归。

\question[3] 牧人驱犊返,\fillin猎马带禽归。

\question[3] \fillin相顾无相识,长歌怀采薇。

\question[3] 相顾无相识,\fillin长歌怀采薇。

\question[3] \fillin饮马渡秋水,水寒风似刀。

\question[3] 饮马渡秋水,\fillin水寒风似刀。

\question[3] \fillin平沙日未没,黯黯见临洮。

\question[3] 平沙日未没,\fillin黯黯见临洮。

\question[3] \fillin昔日长城战,咸言意气高。

\question[3] 昔日长城战,\fillin咸言意气高。

\question[3] \fillin黄尘足今古,白骨乱蓬蒿。

\question[3] 黄尘足今古,\fillin白骨乱蓬蒿。

\question[3] \fillin王濬楼船下益州,金陵王气黯然收。

\question[3] 王濬楼船下益州,\fillin金陵王气黯然收。

\question[3] \fillin千寻铁锁沉江底,一片降幡出石头。

\question[3] 千寻铁锁沉江底,\fillin一片降幡出石头。

\question[3] \fillin人世几回伤往事,山形依旧枕寒流。

\question[3] 人世几回伤往事,\fillin山形依旧枕寒流。

\question[3] \fillin今逢四海为家日,故垒萧萧芦荻秋。

\question[3] 今逢四海为家日,\fillin故垒萧萧芦荻秋。

\question[3] \fillin马穿山径菊初黄,信马悠悠野兴长。

\question[3] 马穿山径菊初黄,\fillin信马悠悠野兴长。

\question[3] \fillin万壑有声含晚籁,数峰无语立斜阳。

\question[3] 万壑有声含晚籁,\fillin数峰无语立斜阳。

\question[3] \fillin棠梨叶落胭脂色,荞麦花开白雪香。

\question[3] 棠梨叶落胭脂色,\fillin荞麦花开白雪香。

\question[3] \fillin何事吟余忽惆怅,村桥原树似吾乡。

\question[3] 何事吟馀忽惆怅,\fillin村桥原树似吾乡。

\question[3] \fillin东风夜放花千树,更吹落、星如雨。

\question[3] 东风夜放花千树,\fillin更吹落、星如雨。

\question[3] \fillin宝马雕车香满路。凤箫声动,玉壶光转,一夜鱼龙舞。

\question[3] 宝马雕车香满路。\fillin凤箫声动,玉壶光转,一夜鱼龙舞。

\question[3] 宝马雕车香满路。凤箫声动,\fillin玉壶光转,一夜鱼龙舞。

\question[3] 宝马雕车香满路。凤箫声动,玉壶光转,\fillin一夜鱼龙舞。

\question[3] \fillin蛾儿雪柳黄金缕,笑语盈盈暗香去。

\question[3] 蛾儿雪柳黄金缕,\fillin笑语盈盈暗香去。

\question[3] \fillin众里寻他千百度,蓦然回首,那人却在,灯火阑珊处。

\question[3] 众里寻他千百度,\fillin蓦然回首,那人却在,灯火阑珊处。

\question[3] 众里寻他千百度,蓦然回首,\fillin那人却在,灯火阑珊处。

\question[3] 众里寻他千百度,蓦然回首,那人却在,\fillin灯火阑珊处。

\question[3] \fillin缺月挂疏桐,漏断人初静。

\question[3] 缺月挂疏桐,\fillin漏断人初静。

\question[3] \fillin谁见幽人独往来,缥缈孤鸿影。

\question[3] 谁见幽人独往来,\fillin缥缈孤鸿影。

\question[3] \fillin惊起却回头,有恨无人省。

\question[3] 惊起却回头,\fillin有恨无人省。

\question[3] \fillin拣尽寒枝不肯栖,寂寞沙洲冷。

\question[3] 拣尽寒枝不肯栖,\fillin寂寞沙洲冷。

\question[3] \fillin槛菊愁烟兰泣露,罗幕轻寒,燕子双飞去。

\question[3] 槛菊愁烟兰泣露,\fillin罗幕轻寒,燕子双飞去。

\question[3] 槛菊愁烟兰泣露,罗幕轻寒,\fillin燕子双飞去。

\question[3] \fillin明月不谙离恨苦,斜光到晓穿朱户。

\question[3] 明月不谙离恨苦,\fillin斜光到晓穿朱户。

\question[3] \fillin昨夜西风凋碧树,独上高楼,望尽天涯路。

\question[3] 昨夜西风凋碧树,\fillin独上高楼,望尽天涯路。

\question[3] 昨夜西风凋碧树,独上高楼,\fillin望尽天涯路。

\question[3] \fillin欲寄彩笺兼尺素,山长水阔知何处?

\question[3] 欲寄彩笺兼尺素,\fillin山长水阔知何处?

\question[3] \fillin水调数声持酒听,午醉醒来愁未醒。送春春去几时回?

\question[3] 水调数声持酒听,\fillin午醉醒来愁未醒。送春春去几时回?

\question[3] 水调数声持酒听,午醉醒来愁未醒。\fillin送春春去几时回?

\question[3] \fillin临晚镜,伤流景,往事后期空记省。

\question[3] 临晚镜,伤流景,\fillin往事后期空记省。

\question[3] \fillin沙上并禽池上暝,云破月来花弄影。

\question[3] 沙上并禽池上暝,\fillin云破月来花弄影。

\question[3] \fillin重重帘幕密遮灯,风不定,人初静,明日落红应满径。

\question[3] 重重帘幕密遮灯,\fillin风不定,人初静,明日落红应满径。

\question[3] 重重帘幕密遮灯,风不定,人初静,\fillin明日落红应满径。

\question[3] \fillin低茅舍,卖酒家,客来旋把朱帘挂。

\question[3] 低茅舍,卖酒家,\fillin客来旋把朱帘挂。

\question[3] \fillin长天落霞,方池睡鸭,老树昏鸦。

\question[3] 长天落霞,\fillin方池睡鸭,老树昏鸦。

\question[3] 长天落霞,方池睡鸭,\fillin老树昏鸦。

\question[3] \fillin几句杜陵诗,-幅王维画。

\question[3] 几句杜陵诗,\fillin-幅王维画。

\question[3] \fillin江汉曾为客,相逢每醉还。

\question[3] 江汉曾为客,\fillin相逢每醉还。

\question[3] \fillin浮云一别后,流水十年间。

\question[3] 浮云一别后,\fillin流水十年间。

\question[3] \fillin欢笑情如旧,萧疏鬓已斑。

\question[3] 欢笑情如旧,\fillin萧疏鬓已斑。

\question[3] \fillin何因不归去?淮上有秋山。

\question[3] 何因不归去?\fillin淮上有秋山。

\question[3] \fillin朝闻游子唱离歌,昨夜微霜初渡河。

\question[3] 朝闻游子唱离歌,\fillin昨夜微霜初渡河。

\question[3] \fillin鸿雁不堪愁里听,云山况是客中过。

\question[3] 鸿雁不堪愁里听,\fillin云山况是客中过。

\question[3] \fillin关城曙色催寒近,御苑砧声向晚多。

\question[3] 关城曙色催寒近,\fillin御苑砧声向晚多。

\question[3] \fillin莫见长安行乐处,空令岁月易蹉跎。

\question[3] 莫见长安行乐处,\fillin空令岁月易蹉跎。

\question[3] \fillin牛渚西江夜,青天无片云。

\question[3] 牛渚西江夜,\fillin青天无片云。

\question[3] \fillin登舟望秋月,空忆谢将军。

\question[3] 登舟望秋月,\fillin空忆谢将军。

\question[3] \fillin余亦能高咏,斯人不可闻。

\question[3] 余亦能高咏,\fillin斯人不可闻。

\question[3] \fillin明朝挂帆去,枫叶落纷纷。

\question[3] 明朝挂帆去,\fillin枫叶落纷纷。

\question[3] \fillin古台摇落后,秋入望乡心。

\question[3] 古台摇落后,\fillin秋入望乡心。

\question[3] \fillin野寺来人少,云峰隔水深。

\question[3] 野寺来人少,\fillin云峰隔水深。

\question[3] \fillin夕阳依旧垒,寒磬满空林。

\question[3] 夕阳依旧垒,\fillin寒磬满空林。

\question[3] \fillin惆怅南朝事,长江独至今。

\question[3] 惆怅南朝事,\fillin长江独至今。

\question[3] \fillin白日登山望烽火,黄昏饮马傍交河。

\question[3] 白日登山望烽火,\fillin黄昏饮马傍交河。

\question[3] \fillin行人刁斗风沙暗,公主琵琶幽怨多。

\question[3] 行人刁斗风沙暗,\fillin公主琵琶幽怨多。

\question[3] \fillin野云万里无城郭,雨雪纷纷连大漠。

\question[3] 野云万里无城郭,\fillin雨雪纷纷连大漠。

\question[3] \fillin胡雁哀鸣夜夜飞,胡儿眼泪双双落。

\question[3] 胡雁哀鸣夜夜飞,\fillin胡儿眼泪双双落。

\question[3] \fillin闻道玉门犹被遮,应将性命逐轻车。

\question[3] 闻道玉门犹被遮,\fillin应将性命逐轻车。

\question[3] \fillin年年战骨埋荒外,空见蒲桃入汉家。

\question[3] 年年战骨埋荒外,\fillin空见蒲桃入汉家。

\question[3] \fillin树老垂缨乱,祠荒向水开。

\question[3] 树老垂缨乱,\fillin祠荒向水开。

\question[3] \fillin偶人经雨踣,古屋为风摧。

\question[3] 偶人经雨踣,\fillin古屋为风摧。

\question[3] \fillin野鸟栖尘坐,渔郎奠竹杯。

\question[3] 野鸟栖尘坐,\fillin渔郎奠竹杯。

\question[3] \fillin欲传《山鬼》曲,无奈《楚辞》哀!

\question[3] 欲传《山鬼》曲,\fillin无奈《楚辞》哀!

\question[3] \fillin红叶晚萧萧,长亭酒一瓢。

\question[3] 红叶晚萧萧,\fillin长亭酒一瓢。

\question[3] \fillin残云归太华,疏雨过中条。

\question[3] 残云归太华,\fillin疏雨过中条。

\question[3] \fillin树色随山迥,河声入海遥。

\question[3] 树色随山迥,\fillin河声入海遥。

\question[3] \fillin帝乡明日到,犹自梦渔樵。

\question[3] 帝乡明日到,\fillin犹自梦渔樵。

\question[3] \fillin何处金衣客,栖栖翠幕中。

\question[3] 何处金衣客,\fillin栖栖翠幕中。

\question[3] \fillin有心惊晓梦,无计啭春风。

\question[3] 有心惊晓梦,\fillin无计啭春风。

\question[3] \fillin漫逐梁间燕,谁巢井上桐。

\question[3] 漫逐梁间燕,\fillin谁巢井上桐。

\question[3] \fillin空将云路翼,缄恨在雕笼。

\question[3] 空将云路翼,\fillin缄恨在雕笼。

\question[3] \fillin岂有千山与万山,别离何易来何难。

\question[3] 岂有千山与万山,\fillin别离何易来何难。

\question[3] \fillin一日一日似流水,他乡故乡空倚阑。

\question[3] 一日一日似流水,\fillin他乡故乡空倚阑。

\question[3] \fillin云补断桥六月雨,松扶古殿三时寒。

\question[3] 云补断桥六月雨,\fillin松扶古殿三时寒。

\question[3] \fillin笋脯茶油新麦饭,几时猿鹤来同餐。

\question[3] 笋脯茶油新麦饭,\fillin几时猿鹤来同餐。

\question[3] \fillin山下兰芽短浸溪,松间沙路净无泥,萧萧暮雨子规啼。

\question[3] 山下兰芽短浸溪,\fillin松间沙路净无泥,萧萧暮雨子规啼。

\question[3] 山下兰芽短浸溪,松间沙路净无泥,\fillin萧萧暮雨子规啼。

\question[3] \fillin谁道人生无再少?门前流水尚能西!休将白发唱黄鸡。

\question[3] 谁道人生无再少?\fillin门前流水尚能西!休将白发唱黄鸡。

\question[3] 谁道人生无再少?门前流水尚能西!\fillin休将白发唱黄鸡。

\question[3] \fillin小雨纤纤风细细,万家杨柳青烟里。

\question[3] 小雨纤纤风细细,\fillin万家杨柳青烟里。

\question[3] \fillin恋树湿花飞不起,愁无比,和春付与东流水。

\question[3] 恋树湿花飞不起,\fillin愁无比,和春付与东流水。

\question[3] 恋树湿花飞不起,愁无比,\fillin和春付与东流水。

\question[3] \fillin九十光阴能有几?金龟解尽留无计。

\question[3] 九十光阴能有几?\fillin金龟解尽留无计。

\question[3] \fillin寄语东阳沽酒市,拼一醉,而今乐事他年泪。

\question[3] 寄语东阳沽酒市,\fillin拼一醉,而今乐事他年泪。

\question[3] 寄语东阳沽酒市,拼一醉,\fillin而今乐事他年泪。

\question[3] \fillin霜后芙蓉落远洲,雁行初过客登楼。

\question[3] 霜后芙蓉落远洲,\fillin雁行初过客登楼。

\question[3] \fillin荒烟平楚苍茫处,极目江南总是秋。

\question[3] 荒烟平楚苍茫处,\fillin极目江南总是秋。

\question[3] \fillin岧峣太华俯咸京,天外三峰削不成。

\question[3] 岧峣太华俯咸京,\fillin天外三峰削不成。

\question[3] \fillin武帝祠前云欲散,仙人掌上雨初晴。

\question[3] 武帝祠前云欲散,\fillin仙人掌上雨初晴。

\question[3] \fillin河山北枕秦关险,驿路西连汉畤平。

\question[3] 河山北枕秦关险,\fillin驿路西连汉畤平。

\question[3] \fillin借问路旁名利客,何如此处学长生。

\question[3] 借问路旁名利客,\fillin何如此处学长生。

\question[3] \fillin故乡飞鸟尚啁啾,何况悲笳出塞愁。

\question[3] 故乡飞鸟尚啁啾,\fillin何况悲笳出塞愁。

\question[3] \fillin青冢埋魂知不返,翠崖遗迹为谁留。

\question[3] 青冢埋魂知不返,\fillin翠崖遗迹为谁留。

\question[3] \fillin玉颜自古为身累,肉食何人与国谋。

\question[3] 玉颜自古为身累,\fillin肉食何人与国谋。

\question[3] \fillin行路至今空叹息,岩花涧草自春秋。

\question[3] 行路至今空叹息,\fillin岩花涧草自春秋。

\question[3] \fillin寒风夕吹易水波,渐离击筑荆卿歌。

\question[3] 寒风夕吹易水波,\fillin渐离击筑荆卿歌。

\question[3] \fillin白衣洒泪当祖路,日落登车去不顾。

\question[3] 白衣洒泪当祖路,\fillin日落登车去不顾。

\question[3] \fillin秦王殿上开地图,舞阳色沮那敢呼。

\question[3] 秦王殿上开地图,\fillin舞阳色沮那敢呼。

\question[3] \fillin手持匕首摘铜柱,事已不成空骂倨。

\question[3] 手持匕首摘铜柱,\fillin事已不成空骂倨。

\question[3] \fillin吁嗟乎! 燕丹寡谋当灭身。

\question[3] 吁嗟乎! \fillin燕丹寡谋当灭身。

\question[3] \fillin田光自刎何足云,惜哉枉杀樊将军。

\question[3] 田光自刎何足云,\fillin惜哉枉杀樊将军。

\question[3] \fillin危楼樽酒赋蒹葭,南望潇湘水一涯。

\question[3] 危楼樽酒赋蒹葭,\fillin南望潇湘水一涯。

\question[3] \fillin云麓半涵青海雾,岸枫遥映赤城霞。

\question[3] 云麓半涵青海雾,\fillin岸枫遥映赤城霞。

\question[3] \fillin双飞日月驱神骏,半缺河山待女娲。

\question[3] 双飞日月驱神骏,\fillin半缺河山待女娲。

\question[3] \fillin学就屠龙空束手,剑锋腾踏绕霜花。

\question[3] 学就屠龙空束手,\fillin剑锋腾踏绕霜花。

\question[3] \fillin东风吹绽红亭树,独上高原愁日暮。

\question[3] 东风吹绽红亭树,\fillin独上高原愁日暮。

\question[3] \fillin可怜骊马啼下尘,吹作游人眼中雾。

\question[3] 可怜骊马啼下尘,\fillin吹作游人眼中雾。

\question[3] \fillin青山渐高日渐低,荒园冻雀一声啼。

\question[3] 青山渐高日渐低,\fillin荒园冻雀一声啼。

\question[3] \fillin三归台畔古碑没,项羽坟头石马嘶。

\question[3] 三归台畔古碑没,\fillin项羽坟头石马嘶。

\end{questions}

\hspace{5cm}

\section{\normalsize{填空题 (本大题共 117 题,满分 234 分)}}
\hspace{1.5cm}
\begin{multicols}{2}
\begin{questions}
\question[2] \fillin天行健,君子以自强不息

\question[2] 天行健,\fillin君子以自强不息

\question[2] \fillin地势坤,君子以厚德载物

\question[2] 地势坤,\fillin君子以厚德载物

\question[2] \fillin三军可夺帅,匹夫不可夺志

\question[2] 三军可夺帅,\fillin匹夫不可夺志

\question[2] \fillin仰不愧于天,俯不怍于人

\question[2] 仰不愧于天,\fillin俯不怍于人

\question[2] \fillin过而不改,是谓过也

\question[2] 过而不改,\fillin是谓过也

\question[2] \fillin不矜细行,终累大德

\question[2] 不矜细行,\fillin终累大德

\question[2] \fillin以人为镜,可以明得失

\question[2] 以人为镜,\fillin可以明得失

\question[2] \fillin飘风不终朝,骤雨不终日

\question[2] 飘风不终朝,\fillin骤雨不终日

\question[2] \fillin大夏将倾,非一木可支

\question[2] 大夏将倾,\fillin非一木可支

\question[2] \fillin信言不美,美言不信

\question[2] 信言不美,\fillin美言不信

\question[2] \fillin剑老无芒,人老无刚

\question[2] 剑老无芒,\fillin人老无刚

\question[2] \fillin前事不忘,后事之师

\question[2] 前事不忘,\fillin后事之师

\question[2] \fillin姜桂之性,到老愈辣

\question[2] 姜桂之性,\fillin到老愈辣

\question[2] \fillin既来之,则安之

\question[2] 既来之,\fillin则安之

\question[2] \fillin差之毫厘,谬以千里

\question[2] 差之毫厘,\fillin谬以千里

\question[2] \fillin屋漏偏逢连夜雨

\question[2] 屋漏\fillin偏逢连夜雨

\question[2] 屋漏偏逢\fillin连夜雨

\question[2] \fillin桃李不言,下自成蹊

\question[2] 桃李不言,\fillin下自成蹊

\question[2] \fillin桃李满天下

\question[2] 桃李\fillin满天下

\question[2] \fillin君子之交淡如水

\question[2] 君子之交\fillin淡如水

\question[2] \fillin小人之交甘若醴

\question[2] 小人之交\fillin甘若醴

\question[2] \fillin炉火灰深到晓温

\question[2] 炉火灰深\fillin到晓温

\question[2] \fillin奇文共欣赏,疑义相与析

\question[2] 奇文共欣赏,\fillin疑义相与析

\question[2] \fillin吟安一个字,捻断数茎须

\question[2] 吟安一个字,\fillin捻断数茎须

\question[2] \fillin一波才动万波随

\question[2] 一波才动\fillin万波随

\question[2] \fillin一语天然万古新

\question[2] 一语天然\fillin万古新

\question[2] \fillin一步一陟一回顾,我脚高时他更高

\question[2] 一步一陟一回顾,\fillin我脚高时他更高

\question[2] \fillin万人丛中一握手,使我衣袖三年香

\question[2] 万人丛中一握手,\fillin使我衣袖三年香

\question[2] \fillin毁誉不干其守,饥寒不累其心

\question[2] 毁誉不干其守,\fillin饥寒不累其心

\question[2] \fillin众怒难犯,专欲难成

\question[2] 众怒难犯,\fillin专欲难成

\question[2] \fillin富贵必从勤苦得

\question[2] 富贵\fillin必从勤苦得

\question[2] 富贵必从\fillin勤苦得

\question[2] \fillin蜗牛角上争何事

\question[2] 蜗牛角上\fillin争何事

\question[2] \fillin愿君学长松,慎勿作桃李

\question[2] 愿君学长松,\fillin慎勿作桃李

\question[2] \fillin心如明镜台

\question[2] 心如\fillin明镜台

\question[2] \fillin此处不留人,自有留人处

\question[2] 此处不留人,\fillin自有留人处

\question[2] \fillin君子坦荡荡,小人长戚戚

\question[2] 君子坦荡荡,\fillin小人长戚戚

\question[2] \fillin老吾老,以及人之老

\question[2] 老吾老,\fillin以及人之老

\question[2] \fillin君子患无德,不患无土

\question[2] 君子患无德,\fillin不患无土

\question[2] \fillin百尺竿头,更进一步

\question[2] 百尺竿头,\fillin更进一步

\question[2] \fillin同是天涯沦落人,相逢何必曾相识

\question[2] 同是天涯沦落人,\fillin相逢何必曾相识

\question[2] \fillin同声相应,同气相求

\question[2] 同声相应,\fillin同气相求

\question[2] \fillin当局者迷,旁观者清

\question[2] 当局者迷,\fillin旁观者清

\question[2] \fillin年年岁岁花相似,岁岁年年人不同

\question[2] 年年岁岁花相似,\fillin岁岁年年人不同

\question[2] \fillin先国家之急而后私仇

\question[2] 先国家之急而\fillin后私仇

\question[2] \fillin防民之口,甚于防川

\question[2] 防民之口,\fillin甚于防川

\question[2] \fillin如闻其声,如见其人

\question[2] 如闻其声,\fillin如见其人

\question[2] \fillin自成一家

\question[2] 自成\fillin一家

\question[2] \fillin江山易改,禀性难移

\question[2] 江山易改,\fillin禀性难移

\question[2] \fillin迅雷不及掩耳

\question[2] 迅雷\fillin不及掩耳

\question[2] 迅雷不及\fillin掩耳

\question[2] \fillin如入宝山空手回

\question[2] 如入宝山\fillin空手回

\question[2] \fillin来而不往非礼也

\question[2] 来而不往\fillin非礼也

\question[2] \fillin行百里者半九十

\question[2] \fillin好语如珠

\question[2] 好语\fillin如珠

\question[2] \fillin近水救不得远火

\question[2] 近水\fillin救不得远火

\question[2] 近水救不得\fillin远火

\question[2] \fillin进可以攻,退可以守

\question[2] 进可以攻,\fillin退可以守

\question[2] \fillin劳于读书,逸于作文

\question[2] 劳于读书,\fillin逸于作文

\question[2] \fillin励精图治

\question[2] 励精\fillin图治

\question[2] \fillin否极泰来

\question[2] 否极\fillin泰来

\end{questions}
\end{multicols}

\hspace{5cm}

\section{\normalsize{多选题 (本大题共 4 题,满分 20 分)}}
\hspace{1.5cm}

\begin{questions}
\question[5] AB

\begin{oneparcheckboxes}
\CorrectChoice 1
\CorrectChoice 2
\choice  3
\choice  4
\end{oneparcheckboxes}

\question[5] ABCD

\begin{oneparcheckboxes}
\CorrectChoice 1
\CorrectChoice 2
\CorrectChoice 3
\CorrectChoice 4
\end{oneparcheckboxes}

\question[5] C

\begin{oneparcheckboxes}
\choice  1
\choice  2
\CorrectChoice 3
\choice  4
\end{oneparcheckboxes}

\question[5] none

\begin{oneparcheckboxes}
\choice  1
\choice  2
\choice  3
\choice  4
\end{oneparcheckboxes}

\end{questions}

\hspace{5cm}

\section{\normalsize{单选题 (本大题共 1 题,满分 20 分)}}
\hspace{1.5cm}

\begin{questions}
\question[20] PigD

\begin{oneparchoices}
\choice  2
\choice  1
\choice  3
\CorrectChoice 4
\end{oneparchoices}

\end{questions}

\hspace{5cm}

\section{\normalsize{主观题 (本大题共 1 题,满分 30 分)}}
\hspace{1.5cm}

\begin{questions}
\question[30] 赏析“可怜无定河边骨,犹是春闺梦里人”

参考答案: \textbf{此句没有直接战争带来的悲惨景象,也没有渲染家人的悲伤情绪,而是匠心独运,把“河边骨”和“春闺梦”联系起来,写闺中妻子不知征人战死,仍然在梦中想见已成白骨的丈夫,使全诗产生震撼心灵的悲剧力量。}

\end{questions}

\end{document}

