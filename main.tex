
\documentclass[12pt, a4paper, addpoints, answers]{exam}
\printanswers

\usepackage{xeCJK}
\setCJKmainfont{SimSun}[BoldFont=SimHei,ItalicFont=KaiTi]

\footer{}{第 \thepage 页 (共 \pageref{LastPage} 页)}{}
\firstpageheader{2023/7/25 00:26:51}{}{姓名:\quad\textbf{李沪纲}}
\runningheader{}{2023/7/25 00:26:51}{}

\usepackage{amssymb}
\usepackage{multicol}
\usepackage{lastpage}

\usepackage{xcolor}

\usepackage[
pdfa=true,
unicode=true,
hidelinks,
pdfauthor={李沪纲},
pdftitle={2023/7/25 00:26:51},
pdfsubject={2023/7/25 00:26:51}]{hyperref}

\usepackage{geometry}
\geometry{a4paper,scale=0.8}

\usepackage{ulem}

\pointformat{(\thepoints)}
\pointname{分}
\checkboxchar{$\square$}
\checkedchar{$\blacksquare$}

\setlength{\parindent}{2em}

\begin{document}

\pagestyle{headandfoot}

\begin{center}
\fbox{\fbox{\parbox{5.5in}{\centering
2023/7/25 00:26:51

命题人:李沪纲

(答题时间 2 分钟,共 \numquestions 题,满分 \numpoints 分)

于 2023/7/25 19:15:56 导出}}}
\end{center}
\vspace{5mm}

\normalsize
\vspace{5mm}
得分:44分\quad 答题用时:2分02秒
\hspace{2cm}

\section{\normalsize{填空题 (本大题共 20 题,满分 80 分)}}
\hspace{1.5cm}

\begin{questions}
\question[4] \fillin ,他乡故乡空倚阑。

标准答案: \textbf{一日一日似流水}   

得分:\textcolor{red}{0分} 

\question[4] 危楼樽酒赋蒹葭,\fillin 。

标准答案: \textbf{南望潇湘水一涯}   

得分:\textcolor{red}{0分} 

\question[4] 映阶碧草自春色,\fillin[隔叶黄鹂空好音] 。

标准答案: \textbf{隔叶黄鹂空好音}   

得分:\textcolor{teal}{4分} 

\question[4] \fillin[双飞燕子几时回] ?夹岸桃花蘸水开。

标准答案: \textbf{双飞燕子几时回}   

得分:\textcolor{teal}{4分} 

\question[4] 寄语东阳沽酒市,\fillin ,而今乐事他年泪。

标准答案: \textbf{拼一醉}   

得分:\textcolor{red}{0分} 

\question[4] 沙上并禽池上暝,\fillin 。

标准答案: \textbf{云破月来花弄影}   

得分:\textcolor{red}{0分} 

\question[4] \fillin[忽见陌头杨柳色] ,悔教夫婿觅封侯。

标准答案: \textbf{忽见陌头杨柳色}   

得分:\textcolor{teal}{4分} 

\question[4] 欢笑情如旧,\fillin 。

标准答案: \textbf{萧疏鬓已斑}   

得分:\textcolor{red}{0分} 

\question[4] \fillin[忽见陌头杨柳色] ,悔教夫婿觅封侯。

标准答案: \textbf{忽见陌头杨柳色}   

得分:\textcolor{teal}{4分} 

\question[4] \fillin ,空见蒲桃入汉家。

标准答案: \textbf{年年战骨埋荒外}   

得分:\textcolor{red}{0分} 

\question[4] \fillin[气蒸云梦泽] ,波撼岳阳城。

标准答案: \textbf{气蒸云梦泽}   

得分:\textcolor{teal}{4分} 

\question[4] \fillin ,别离何易来何难。

标准答案: \textbf{岂有千山与万山}   

得分:\textcolor{red}{0分} 

\question[4] \fillin ,日落登车去不顾。

标准答案: \textbf{白衣洒泪当祖路}   

得分:\textcolor{red}{0分} 

\question[4] \fillin ,渔郎奠竹杯。

标准答案: \textbf{野鸟栖尘坐}   

得分:\textcolor{red}{0分} 

\question[4] \fillin ,相逢每醉还。

标准答案: \textbf{江汉曾为客}   

得分:\textcolor{red}{0分} 

\question[4] 气蒸云梦泽,\fillin[波撼岳阳城] 。

标准答案: \textbf{波撼岳阳城}   

得分:\textcolor{teal}{4分} 

\question[4] \fillin ,公主琵琶幽怨多。

标准答案: \textbf{行人刁斗风沙暗}   

得分:\textcolor{red}{0分} 

\question[4] \fillin[朱楼四面勾疏箔] ,卧看千山急雨来。

标准答案: \textbf{朱楼四面钩疏箔}   

得分:\textcolor{red}{0分} 

\question[4] \fillin[黄尘足今古] ,白骨乱蓬蒿。

标准答案: \textbf{黄尘足今古}   

得分:\textcolor{teal}{4分} 

\question[4] 夕阳依旧垒,\fillin 。

标准答案: \textbf{寒磬满空林}   

得分:\textcolor{red}{0分} 
\end{questions}

\hspace{5cm}

\section{\normalsize{填空题 (本大题共 20 题,满分 20 分)}}
\hspace{1.5cm}
\begin{multicols}{2}
\begin{questions}
\question[1] 不矜细行,\fillin[终累大德] 

标准答案: \textbf{终累大德}   

得分:\textcolor{teal}{1分} 

\question[1] \fillin[迅雷] 不及掩耳

标准答案: \textbf{迅雷}   

得分:\textcolor{teal}{1分} 

\question[1] 迅雷不及\fillin[掩耳] 

标准答案: \textbf{掩耳}   

得分:\textcolor{teal}{1分} 

\question[1] \fillin[君子之交] 淡如水

标准答案: \textbf{君子之交}   

得分:\textcolor{teal}{1分} 

\question[1] \fillin[桃李不言] ,下自成蹊

标准答案: \textbf{桃李不言}   

得分:\textcolor{teal}{1分} 

\question[1] \fillin[仰不愧于天] ,俯不怍于人

标准答案: \textbf{仰不愧于天}   

得分:\textcolor{teal}{1分} 

\question[1] 君子患无德,\fillin[不患无士] 

标准答案: \textbf{不患无土}   

得分:\textcolor{red}{0分} 

\question[1] 仰不愧于天,\fillin[俯不愧于人] 

标准答案: \textbf{俯不怍于人}   

得分:\textcolor{red}{0分} 

\question[1] 以人为镜,\fillin[可以明得失] 

标准答案: \textbf{可以明得失}   

得分:\textcolor{teal}{1分} 

\question[1] 桃李\fillin[满天下] 

标准答案: \textbf{满天下}   

得分:\textcolor{teal}{1分} 

\question[1] \fillin[前事不忘] ,后事之师

标准答案: \textbf{前事不忘}   

得分:\textcolor{teal}{1分} 

\question[1] 地势坤,\fillin[君子以厚德载物] 

标准答案: \textbf{君子以厚德载物}   

得分:\textcolor{teal}{1分} 

\question[1] \fillin ,专欲难成

标准答案: \textbf{众怒难犯}   

得分:\textcolor{red}{0分} 

\question[1] \fillin ,终累大德

标准答案: \textbf{不矜细行}   

得分:\textcolor{red}{0分} 

\question[1] 一步一陟一回顾,\fillin 

标准答案: \textbf{我脚高时他更高}   

得分:\textcolor{red}{0分} 

\question[1] \fillin ,则安之

标准答案: \textbf{既来之}   

得分:\textcolor{red}{0分} 

\question[1] \fillin ,匹夫不可夺志

标准答案: \textbf{三军可夺帅}   

得分:\textcolor{red}{0分} 

\question[1] 富贵必从\fillin 

标准答案: \textbf{勤苦得}   

得分:\textcolor{red}{0分} 

\question[1] 年年岁岁花相似,\fillin 

标准答案: \textbf{岁岁年年人不同}   

得分:\textcolor{red}{0分} 

\question[1] \fillin ,美言不信

标准答案: \textbf{信言不美}   

得分:\textcolor{red}{0分} 
\end{questions}
\end{multicols}

\end{document}

