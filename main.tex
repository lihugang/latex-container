
\documentclass[12pt, a4paper, addpoints]{exam}
\usepackage{xeCJK}
\setCJKmainfont{SimSun}[BoldFont=SimHei,ItalicFont=KaiTi]

\footer{}{第 \thepage 页 (共 \pageref{LastPage} 页)}{}
\firstpageheader{2023/7/18 05:43:09}{}{姓名:\qquad\qquad}
\runningheader{}{2023/7/18 05:43:09}{}

\usepackage{multicol}
\usepackage{lastpage}

\usepackage[
pdfa=true,
unicode=true,
hidelinks,
pdfauthor={李沪纲},
pdftitle={2023/7/18 05:43:09},
pdfsubject={2023/7/18 05:43:09}]{hyperref}

\usepackage{geometry}
\geometry{a4paper,scale=0.8}

\usepackage{ulem}

% \usepackage{fancyhdr}
% \pagestyle{fancy}
% \fancyhead[L]{2023/7/18 05:43:09}
% \fancyhead[R]{班级:\qquad 姓名:\qquad\qquad}
% \fancyfoot[C]{第 \thepage 页 (共 \pageref{LastPage} 页)}

\pointformat{(\thepoints)}
\pointname{分}

\setlength{\parindent}{2em}

\begin{document}

\pagestyle{headandfoot}

\begin{center}
\fbox{\fbox{\parbox{5.5in}{\centering
2023/7/18 05:43:09

命题人:李沪纲

(答题时间 99 分钟,共 120 题,满分 120 分)

于 2023/7/19 09:30:34 导出}}}
\end{center}
\vspace{5mm}

\normalsize
\vspace{5mm}

\section{\normalsize{填空题 (本大题共 100 题,满分 100 分)}}
\hspace{1.5cm}

\begin{questions}
\question[1] 秦王殿上开地图,\uline{\qquad\qquad\qquad}。

\question[1] 空将云路翼,\uline{\qquad\qquad\qquad}。

\question[1] \uline{\qquad\qquad\qquad},仙人掌上雨初晴。

\question[1] 偶人经雨踣,\uline{\qquad\qquad\qquad}。

\question[1] 白日登山望烽火,\uline{\qquad\qquad\qquad}。

\question[1] \uline{\qquad\qquad\qquad},公主琵琶幽怨多。

\question[1] 水调数声持酒听,午醉醒来愁未醒。\uline{\qquad\qquad\qquad}?

\question[1] 诗家清景在新春,\uline{\qquad\qquad\qquad}。

\question[1] 人世几回伤往事,\uline{\qquad\qquad\qquad}。

\question[1] 槛菊愁烟兰泣露,\uline{\qquad\qquad\qquad},燕子双飞去。

\question[1] 岧峣太华俯咸京,\uline{\qquad\qquad\qquad}。

\question[1] 寒山转苍翠,\uline{\qquad\qquad\qquad}。

\question[1] 岂有千山与万山,\uline{\qquad\qquad\qquad}。

\question[1] \uline{\qquad\qquad\qquad},更吹落、星如雨。

\question[1] 树老垂缨乱,\uline{\qquad\qquad\qquad}。

\question[1] \uline{\qquad\qquad\qquad},愁无比,和春付与东流水。

\question[1] 昔日长城战,\uline{\qquad\qquad\qquad}。

\question[1] 谁见幽人独往来,\uline{\qquad\qquad\qquad}。

\question[1] \uline{\qquad\qquad\qquad},枫叶落纷纷。

\question[1] \uline{\qquad\qquad\qquad},两朝开济老臣心。

\question[1] 谁见幽人独往来,\uline{\qquad\qquad\qquad}。

\question[1] \uline{\qquad\qquad\qquad},岩花涧草自春秋。

\question[1] \uline{\qquad\qquad\qquad},陂塘分出几泉清?

\question[1] 若待上林花似锦,\uline{\qquad\qquad\qquad}。

\question[1] 恋树湿花飞不起,\uline{\qquad\qquad\qquad},和春付与东流水。

\question[1] 水落鱼梁浅,\uline{\qquad\qquad\qquad}。

\question[1] 漫逐梁间燕,\uline{\qquad\qquad\qquad}。

\question[1] \uline{\qquad\qquad\qquad},波撼岳阳城。

\question[1] 槛菊愁烟兰泣露,\uline{\qquad\qquad\qquad},燕子双飞去。

\question[1] 白日登山望烽火,\uline{\qquad\qquad\qquad}。

\question[1] \uline{\qquad\qquad\qquad},渔郎奠竹杯。

\question[1] \uline{\qquad\qquad\qquad},长使英雄泪满襟。

\question[1] \uline{\qquad\qquad\qquad}?夹岸桃花蘸水开。

\question[1] \uline{\qquad\qquad\qquad},隔叶黄鹂空好音。

\question[1] \uline{\qquad\qquad\qquad},绿柳才黄半未匀。

\question[1] 昔日长城战,\uline{\qquad\qquad\qquad}。

\question[1] \uline{\qquad\qquad\qquad},信马悠悠野兴长。

\question[1] 双飞燕子几时回?\uline{\qquad\qquad\qquad}。

\question[1] 拣尽寒枝不肯栖,\uline{\qquad\qquad\qquad}。

\question[1] 小雨纤纤风细细,\uline{\qquad\qquad\qquad}。

\question[1] \uline{\qquad\qquad\qquad},长江独至今。

\question[1] 月下飞天镜,\uline{\qquad\qquad\qquad}。

\question[1] 欲传《山鬼》曲,\uline{\qquad\qquad\qquad}!

\question[1] 谁见幽人独往来,\uline{\qquad\qquad\qquad}。

\question[1] 八月湖水平,\uline{\qquad\qquad\qquad}。

\question[1] \uline{\qquad\qquad\qquad},斯人不可闻。

\question[1] 春雨断桥人不度,\uline{\qquad\qquad\qquad}。

\question[1] 余亦能高咏,\uline{\qquad\qquad\qquad}。

\question[1] 惆怅南朝事,\uline{\qquad\qquad\qquad}。

\question[1] \uline{\qquad\qquad\qquad},云山况是客中过。

\question[1] \uline{\qquad\qquad\qquad},斯人不可闻。

\question[1] \uline{\qquad\qquad\qquad},青草池塘处处蛙。

\question[1] 若待上林花似锦,\uline{\qquad\qquad\qquad}。

\question[1] 忽见陌头杨柳色,\uline{\qquad\qquad\qquad}。

\question[1] \uline{\qquad\qquad\qquad},云生结海楼。

\question[1] \uline{\qquad\qquad\qquad},午醉醒来愁未醒。送春春去几时回?

\question[1] \uline{\qquad\qquad\qquad},斯人不可闻。

\question[1] 映阶碧草自春色,\uline{\qquad\qquad\qquad}。

\question[1] \uline{\qquad\qquad\qquad},陂塘分出几泉清?

\question[1] 夕阳依旧垒,\uline{\qquad\qquad\qquad}。

\question[1] \uline{\qquad\qquad\qquad},黄昏饮马傍交河。

\question[1] 欲传《山鬼》曲,\uline{\qquad\qquad\qquad}!

\question[1] 空将云路翼,\uline{\qquad\qquad\qquad}。

\question[1] \uline{\qquad\qquad\qquad},水寒风似刀。

\question[1] 太乙近天都,\uline{\qquad\qquad\qquad}。

\question[1] \uline{\qquad\qquad\qquad},北风吹起数声雷。

\question[1] 山随平野尽,\uline{\qquad\qquad\qquad}。

\question[1] 王濬楼船下益州,\uline{\qquad\qquad\qquad}。

\question[1] 余亦能高咏,\uline{\qquad\qquad\qquad}。

\question[1] \uline{\qquad\qquad\qquad},更吹落、星如雨。

\question[1] 古台摇落后,\uline{\qquad\qquad\qquad}。

\question[1] \uline{\qquad\qquad\qquad},长歌怀采薇。

\question[1] \uline{\qquad\qquad\qquad},绿柳才黄半未匀。

\question[1] \uline{\qquad\qquad\qquad},无奈《楚辞》哀!

\question[1] \uline{\qquad\qquad\qquad},云破月来花弄影。

\question[1] 野鸟栖尘坐,\uline{\qquad\qquad\qquad}。

\question[1] \uline{\qquad\qquad\qquad},山长水阔知何处?

\question[1] 白日登山望烽火,\uline{\qquad\qquad\qquad}。

\question[1] 年年战骨埋荒外,\uline{\qquad\qquad\qquad}。

\question[1] \uline{\qquad\qquad\qquad},长使英雄泪满襟。

\question[1] 月下飞天镜,\uline{\qquad\qquad\qquad}。

\question[1] 忽见陌头杨柳色,\uline{\qquad\qquad\qquad}。

\question[1] \uline{\qquad\qquad\qquad},御苑砧声向晚多。

\question[1] 残云归太华,\uline{\qquad\qquad\qquad}。

\question[1] \uline{\qquad\qquad\qquad},山形依旧枕寒流。

\question[1] 树树皆秋色,\uline{\qquad\qquad\qquad}。

\question[1] \uline{\qquad\qquad\qquad},波撼岳阳城。

\question[1] 闻道玉门犹被遮,\uline{\qquad\qquad\qquad}。

\question[1] \uline{\qquad\qquad\qquad}。凤箫声动,玉壶光转,一夜鱼龙舞。

\question[1] 八月湖水平,\uline{\qquad\qquad\qquad}。

\question[1] 古台摇落后,\uline{\qquad\qquad\qquad}。

\question[1] 树老垂缨乱,\uline{\qquad\qquad\qquad}。

\question[1] \uline{\qquad\qquad\qquad}。凤箫声动,玉壶光转,一夜鱼龙舞。

\question[1] 夕阳依旧垒,\uline{\qquad\qquad\qquad}。

\question[1] 马穿山径菊初黄,\uline{\qquad\qquad\qquad}。

\question[1] 山随平野尽,\uline{\qquad\qquad\qquad}。

\question[1] \uline{\qquad\qquad\qquad},往来成古今。

\question[1] \uline{\qquad\qquad\qquad},狂歌五柳前。

\question[1] \uline{\qquad\qquad\qquad},青霭入看无。

\question[1] \uline{\qquad\qquad\qquad},水寒风似刀。

\end{questions}

\hspace{5cm}

\section{\normalsize{填空题 (本大题共 20 题,满分 20 分)}}
\hspace{1.5cm}
\begin{multicols}{2}
\begin{questions}
\question[1] \uline{\qquad\qquad\qquad},使我衣袖三年香

\question[1] \uline{\qquad\qquad\qquad}如珠

\question[1] \uline{\qquad\qquad\qquad},到老愈辣

\question[1] 愿君学长松,\uline{\qquad\qquad\qquad}

\question[1] \uline{\qquad\qquad\qquad},疑义相与析

\question[1] \uline{\qquad\qquad\qquad},更进一步

\question[1] \uline{\qquad\qquad\qquad}而后私仇

\question[1] 过而不改,\uline{\qquad\qquad\qquad}

\question[1] 大夏将倾,\uline{\qquad\qquad\qquad}

\question[1] 仰不愧于天,\uline{\qquad\qquad\qquad}

\question[1] \uline{\qquad\qquad\qquad}如珠

\question[1] 此处不留人,\uline{\qquad\qquad\qquad}

\question[1] \uline{\qquad\qquad\qquad}到晓温

\question[1] 愿君学长松,\uline{\qquad\qquad\qquad}

\question[1] \uline{\qquad\qquad\qquad}争何事

\question[1] \uline{\qquad\qquad\qquad}淡如水

\question[1] 老吾老,\uline{\qquad\qquad\qquad}

\question[1] 愿君学长松,\uline{\qquad\qquad\qquad}

\question[1] \uline{\qquad\qquad\qquad},终累大德

\question[1] \uline{\qquad\qquad\qquad}万波随

\end{questions}
\end{multicols}

\end{document}

