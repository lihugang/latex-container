
\documentclass[12pt, a4paper, addpoints]{exam}
\usepackage{xeCJK}
\setCJKmainfont{SimSun}[BoldFont=SimHei,ItalicFont=KaiTi]

\footer{}{第 \thepage 页 (共 \pageref{LastPage} 页)}{}
\firstpageheader{2023/7/25 00:26:51}{}{姓名:\quad\textbf{李沪纲}}
\runningheader{}{2023/7/25 00:26:51}{}

\usepackage{multicol}
\usepackage{lastpage}

\usepackage[
pdfa=true,
unicode=true,
hidelinks,
pdfauthor={李沪纲},
pdftitle={2023/7/25 00:26:51},
pdfsubject={2023/7/25 00:26:51}]{hyperref}

\usepackage{geometry}
\geometry{a4paper,scale=0.8}

\usepackage{ulem}

\pointformat{(\thepoints)}
\pointname{分}

\setlength{\parindent}{2em}

\begin{document}

\pagestyle{headandfoot}

\begin{center}
\fbox{\fbox{\parbox{5.5in}{\centering
2023/7/25 00:26:51

命题人:李沪纲

(答题时间 2 分钟,共 40 题,满分 100 分)

于 2023/12/29 05:56:06 导出}}}
\end{center}
\vspace{5mm}

\normalsize
\vspace{5mm}

\section{\normalsize{填空题 (本大题共 20 题,满分 80 分)}}
\hspace{1.5cm}

\begin{questions}
\question[4] \uline{\qquad\qquad\qquad},空忆谢将军。

我的答案:

\question[4] 槛菊愁烟兰泣露,罗幕轻寒,\uline{\qquad\qquad\qquad}。

我的答案:

\question[4] 东风夜放花千树,\uline{\qquad\qquad\qquad}。

我的答案:

\question[4] \uline{\qquad\qquad\qquad},有恨无人省。

我的答案:

\question[4] 东风吹绽红亭树,\uline{\qquad\qquad\qquad}。

我的答案:

\question[4] 朝闻游子唱离歌,\uline{\qquad\qquad\qquad}。

我的答案:

\question[4] \uline{\qquad\qquad\qquad},缥缈孤鸿影。

我的答案:

\question[4] \uline{\qquad\qquad\qquad},别离何易来何难。

我的答案:

\question[4] 太乙近天都,\uline{\qquad\qquad\qquad}。

我的答案:

\question[4] 有心惊晓梦,\uline{\qquad\qquad\qquad}。

我的答案:

\question[4] 今逢四海为家日,\uline{\qquad\qquad\qquad}。

我的答案:

\question[4] 黄梅时节家家雨,\uline{\qquad\qquad\qquad}。

我的答案:

\question[4] 登舟望秋月,\uline{\qquad\qquad\qquad}。

我的答案:

\question[4] 千寻铁锁沉江底,\uline{\qquad\qquad\qquad}。

我的答案:

\question[4] \uline{\qquad\qquad\qquad},惜哉枉杀樊将军。

我的答案:

\question[4] 行路至今空叹息,\uline{\qquad\qquad\qquad}。

我的答案:

\question[4] 昨夜西风凋碧树,独上高楼,\uline{\qquad\qquad\qquad}。

我的答案:

\question[4] 河山北枕秦关险,\uline{\qquad\qquad\qquad}。

我的答案:

\question[4] \uline{\qquad\qquad\qquad},村桥原树似吾乡。

我的答案:

\question[4] \uline{\qquad\qquad\qquad},缥缈孤鸿影。

我的答案:

\end{questions}

\hspace{5cm}

\section{\normalsize{填空题 (本大题共 20 题,满分 20 分)}}
\hspace{1.5cm}
\begin{multicols}{2}
\begin{questions}
\question[1] \uline{\qquad\qquad\qquad},美言不信

我的答案:

\question[1] 年年岁岁花相似,\uline{\qquad\qquad\qquad}

我的答案:

\question[1] \uline{\qquad\qquad\qquad},使我衣袖三年香

我的答案:

\question[1] 君子患无德,\uline{\qquad\qquad\qquad}

我的答案:

\question[1] 不矜细行,\uline{\qquad\qquad\qquad}

我的答案:

\question[1] 万人丛中一握手,\uline{\qquad\qquad\qquad}

我的答案:

\question[1] \uline{\qquad\qquad\qquad},俯不怍于人

我的答案:

\question[1] \uline{\qquad\qquad\qquad}偏逢连夜雨

我的答案:

\question[1] 百尺竿头,\uline{\qquad\qquad\qquad}

我的答案:

\question[1] 老吾老,\uline{\qquad\qquad\qquad}

我的答案:

\question[1] \uline{\qquad\qquad\qquad},君子以厚德载物

我的答案:

\question[1] 桃李不言,\uline{\qquad\qquad\qquad}

我的答案:

\question[1] 不矜细行,\uline{\qquad\qquad\qquad}

我的答案:

\question[1] 仰不愧于天,\uline{\qquad\qquad\qquad}

我的答案:

\question[1] 自成\uline{\qquad\qquad\qquad}

我的答案:

\question[1] \uline{\qquad\qquad\qquad}明镜台

我的答案:

\question[1] \uline{\qquad\qquad\qquad},可以明得失

我的答案:

\question[1] 炉火灰深\uline{\qquad\qquad\qquad}

我的答案:

\question[1] 同声相应,\uline{\qquad\qquad\qquad}

我的答案:

\question[1] \uline{\qquad\qquad\qquad}到晓温

我的答案:

\end{questions}
\end{multicols}

\end{document}

