
\documentclass[12pt, a4paper, addpoints, answers]{exam}
\printanswers

\usepackage{xeCJK}
\setCJKmainfont{SimSun}[BoldFont=SimHei,ItalicFont=KaiTi]

\footer{}{第 \thepage 页 (共 \pageref{LastPage} 页)}{}
\firstpageheader{Test}{}{姓名:\quad\textbf{李沪纲}}
\runningheader{}{Test}{}

\usepackage{amssymb}
\usepackage{multicol}
\usepackage{lastpage}

\usepackage{xcolor}

\usepackage[
pdfa=true,
unicode=true,
hidelinks,
pdfauthor={李沪纲},
pdftitle={Test},
pdfsubject={Test}]{hyperref}

\usepackage{geometry}
\geometry{a4paper,scale=0.8}

\usepackage{ulem}

\pointformat{(\thepoints)}
\pointname{分}
\checkboxchar{$\square$}
\checkedchar{$\blacksquare$}

\setlength{\parindent}{2em}

\begin{document}

\pagestyle{headandfoot}

\begin{center}
\fbox{\fbox{\parbox{5.5in}{\centering
Test

命题人:李沪纲

(答题时间 2 分钟,共 \numquestions 题,满分 \numpoints 分)

于 2023/7/25 13:27:45 导出}}}
\end{center}
\vspace{5mm}

\normalsize
\vspace{5mm}
得分:判题未完成\quad 答题用时:2分02秒}

\hspace{2cm}

\section{\normalsize{填空题 (本大题共 20 题,满分 60 分)}}
\hspace{1.5cm}

\begin{questions}
\question[3] 东风吹绽红亭树,\fillin 。

标准答案: \textbf{独上高原愁日暮}

\question[3] 寒风夕吹易水波,\fillin 。

标准答案: \textbf{渐离击筑荆卿歌}

\question[3] 明月不谙离恨苦,\fillin 。

标准答案: \textbf{斜光到晓穿朱户}

\question[3] \fillin ,城中增暮寒。

标准答案: \textbf{林表明霁色}

\question[3] \fillin ,村桥原树似吾乡。

标准答案: \textbf{何事吟余忽惆怅}

\question[3] \fillin ,缄恨在雕笼。

标准答案: \textbf{空将云路翼}

\question[3] 今逢四海为家日,\fillin 。

标准答案: \textbf{故垒萧萧芦荻秋}

\question[3] 水落鱼梁浅,\fillin[天寒梦泽深] 。

标准答案: \textbf{天寒梦泽深}

\question[3] \fillin ,相逢每醉还。

标准答案: \textbf{江汉曾为客}

\question[3] 惊起却回头,\fillin[有恨无人省] 。

标准答案: \textbf{有恨无人省}

\question[3] 欲寄彩笺兼尺素,\fillin ?

标准答案: \textbf{山长水阔知何处}

\question[3] 莫见长安行乐处,\fillin 。

标准答案: \textbf{空令岁月易蹉跎}

\question[3] 明月不谙离恨苦,\fillin 。

标准答案: \textbf{斜光到晓穿朱户}

\question[3] 笋脯茶油新麦饭,\fillin 。

标准答案: \textbf{几时猿鹤来同餐}

\question[3] 平沙日未没,\fillin 。

标准答案: \textbf{黯黯见临洮}

\question[3] \fillin ,绿柳才黄半未匀。

标准答案: \textbf{诗家清景在新春}

\question[3] 重重帘幕密遮灯,\fillin ,明日落红应满径。

标准答案: \textbf{风不定,人初静}

\question[3] 借问路旁名利客,\fillin 。

标准答案: \textbf{何如此处学长生}

\question[3] \fillin ,风不定,人初静,明日落红应满径。

标准答案: \textbf{重重帘幕密遮灯}

\question[3] 寄语东阳沽酒市,拼一醉,\fillin 。

标准答案: \textbf{而今乐事他年泪}
\end{questions}

\hspace{5cm}

\section{\normalsize{填空题 (本大题共 10 题,满分 20 分)}}
\hspace{1.5cm}
\begin{multicols}{2}
\begin{questions}
\question[2] \fillin ,岁岁年年人不同

标准答案: \textbf{年年岁岁花相似}

\question[2] \fillin ,专欲难成

标准答案: \textbf{众怒难犯}

\question[2] \fillin ,可以明得失

标准答案: \textbf{以人为镜}

\question[2] 毁誉不干其守,\fillin 

标准答案: \textbf{饥寒不累其心}

\question[2] \fillin 明镜台

标准答案: \textbf{心如}

\question[2] 富贵\fillin 勤苦得

标准答案: \textbf{必从}

\question[2] 蜗牛角上\fillin 

标准答案: \textbf{争何事}

\question[2] 富贵\fillin 勤苦得

标准答案: \textbf{必从}

\question[2] 老吾老,\fillin 

标准答案: \textbf{以及人之老}

\question[2] \fillin ,美言不信

标准答案: \textbf{信言不美}
\end{questions}
\end{multicols}

\hspace{5cm}

\section{\normalsize{多选题 (本大题共 4 题,满分 20 分)}}
\hspace{1.5cm}

\begin{questions}
\question[5] AB

\begin{oneparcheckboxes}
\CorrectChoice 1
\CorrectChoice 2
\choice  3
\choice  4
\end{oneparcheckboxes}

标准答案: \textbf{1,2}

\question[5] C

\begin{oneparcheckboxes}
\choice  1
\choice  2
\CorrectChoice 3
\choice  4
\end{oneparcheckboxes}

标准答案: \textbf{3}

\question[5] AB

\begin{oneparcheckboxes}
\CorrectChoice 1
\CorrectChoice 2
\choice  3
\choice  4
\end{oneparcheckboxes}

标准答案: \textbf{1,2}

\question[5] AB

\begin{oneparcheckboxes}
\CorrectChoice 1
\CorrectChoice 2
\choice  3
\choice  4
\end{oneparcheckboxes}

标准答案: \textbf{1,2}

\end{questions}

\hspace{5cm}

\section{\normalsize{单选题 (本大题共 1 题,满分 20 分)}}
\hspace{1.5cm}

\begin{questions}
\question[20] PigD

\begin{oneparchoices}
\choice  2
\choice  1
\choice  3
\CorrectChoice 4
\end{oneparchoices}

标准答案: \textbf{4}

\end{questions}

\hspace{5cm}

\section{\normalsize{主观题 (本大题共 1 题,满分 30 分)}}
\hspace{1.5cm}

\begin{questions}
\question[30] 赏析“可怜无定河边骨,犹是春闺梦里人”

我的答案:好

参考答案: \textbf{此句没有直接战争带来的悲惨景象,也没有渲染家人的悲伤情绪,而是匠心独运,把“河边骨”和“春闺梦”联系起来,写闺中妻子不知征人战死,仍然在梦中想见已成白骨的丈夫,使全诗产生震撼心灵的悲剧力量。}

\end{questions}

\end{document}

