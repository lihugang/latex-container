
\documentclass[12pt, a4paper, addpoints]{exam}
\usepackage{xeCJK}
\setCJKmainfont{SimSun}[BoldFont=SimHei,ItalicFont=KaiTi]

\footer{}{第 \thepage 页 (共 \pageref{LastPage} 页)}{}
\firstpageheader{诗句默写 / 2023古诗文}{}{姓名:\qquad\qquad}
\runningheader{}{诗句默写 / 2023古诗文}{}

\usepackage{lastpage}

\usepackage[
pdfa=true,
unicode=true,
hidelinks,
pdfauthor={李沪纲},
pdftitle={诗句默写 / 2023古诗文},
pdfsubject={诗句默写 / 2023古诗文}]{hyperref}

\usepackage{geometry}
\geometry{a4paper,scale=0.8}

\usepackage{ulem}

% \usepackage{fancyhdr}
% \pagestyle{fancy}
% \fancyhead[L]{诗句默写 / 2023古诗文}
% \fancyhead[R]{班级:\qquad 姓名:\qquad\qquad}
% \fancyfoot[C]{第 \thepage 页 (共 \pageref{LastPage} 页)}

\pointformat{(\thepoints)}
\pointname{分}

\setlength{\parindent}{2em}

\begin{document}

\pagestyle{headandfoot}

\begin{center}
\fbox{\fbox{\parbox{5.5in}{\centering
诗句默写 / 2023古诗文

命题人:李沪纲

(答题时间 10 分钟,共 50 题,满分 100 分)

于 2023/7/17 16:43:20 导出}}}
\end{center}
\vspace{5mm}

\normalsize
\vspace{5mm}

\section{\normalsize{填空题 (本大题共 50 题,满分 100 分)}}
\hspace{1.5cm}
\begin{questions}
\question[2] \uline{\qquad\qquad\qquad\qquad},江入大荒流。

\question[2] \uline{\qquad\qquad\qquad\qquad},万里送行舟。

\question[2] 树树皆秋色,\uline{\qquad\qquad\qquad\qquad}。

\question[2] \uline{\qquad\qquad\qquad\qquad},笑语盈盈暗香去。

\question[2] 朱楼四面钩疏箔,\uline{\qquad\qquad\qquad\qquad}。

\question[2] 何事吟馀忽惆怅,\uline{\qquad\qquad\qquad\qquad}。

\question[2] 拣尽寒枝不肯栖,\uline{\qquad\qquad\qquad\qquad}。

\question[2] 明月不谙离恨苦,\uline{\qquad\qquad\qquad\qquad}。

\question[2] 人事有代谢,\uline{\qquad\qquad\qquad\qquad}。

\question[2] 复值接舆醉,\uline{\qquad\qquad\qquad\qquad}。

\question[2] \uline{\qquad\qquad\qquad\qquad},小舟撑出柳阴来。

\question[2] 黄梅时节家家雨,\uline{\qquad\qquad\qquad\qquad}。

\question[2] \uline{\qquad\qquad\qquad\qquad},徙倚欲何依。

\question[2] 宝马雕车香满路。凤箫声动,玉壶光转,\uline{\qquad\qquad\qquad\qquad}。

\question[2] \uline{\qquad\qquad\qquad\qquad},蓦然回首,那人却在,灯火阑珊处。

\question[2] 白鸟一双临水立,\uline{\qquad\qquad\qquad\qquad}。

\question[2] 惊起却回头,\uline{\qquad\qquad\qquad\qquad}。

\question[2] \uline{\qquad\qquad\qquad\qquad},潮退渔船阁岸斜。

\question[2] \uline{\qquad\qquad\qquad\qquad},黯黯见临洮。

\question[2] \uline{\qquad\qquad\qquad\qquad},长歌怀采薇。

\question[2] \uline{\qquad\qquad\qquad\qquad},山长水阔知何处?

\question[2] 渡远荆门外,\uline{\qquad\qquad\qquad\qquad}。

\question[2] 倚杖柴门外,\uline{\qquad\qquad\qquad\qquad}。

\question[2] 牧人驱犊返,\uline{\qquad\qquad\qquad\qquad}。

\question[2] 诗家清景在新春,\uline{\qquad\qquad\qquad\qquad}。

\question[2] \uline{\qquad\qquad\qquad\qquad},天寒梦泽深。

\question[2] \uline{\qquad\qquad\qquad\qquad},村桥原树似吾乡。

\question[2] \uline{\qquad\qquad\qquad\qquad},北风吹起数声雷。

\question[2] 闺中少妇不知愁,\uline{\qquad\qquad\qquad\qquad}。

\question[2] \uline{\qquad\qquad\qquad\qquad},悔教夫婿觅封侯。

\question[2] 海浪如云去却回,\uline{\qquad\qquad\qquad\qquad}。

\question[2] \uline{\qquad\qquad\qquad\qquad},猎马带禽归。

\question[2] 江山留胜迹,\uline{\qquad\qquad\qquad\qquad}。

\question[2] \uline{\qquad\qquad\qquad\qquad},城中增暮寒。

\question[2] 林表明霁色,\uline{\qquad\qquad\qquad\qquad}。

\question[2] \uline{\qquad\qquad\qquad\qquad},云生结海楼。

\question[2] \uline{\qquad\qquad\qquad\qquad},山形依旧枕寒流。

\question[2] 山随平野尽,\uline{\qquad\qquad\qquad\qquad}。

\question[2] \uline{\qquad\qquad\qquad\qquad}。凤箫声动,玉壶光转,一夜鱼龙舞。

\question[2] 谁见幽人独往来,\uline{\qquad\qquad\qquad\qquad}。

\question[2] \uline{\qquad\qquad\qquad\qquad},读罢泪沾襟。

\question[2] \uline{\qquad\qquad\qquad\qquad},我辈复登临。

\question[2] 饮马渡秋水,\uline{\qquad\qquad\qquad\qquad}。

\question[2] 宝马雕车香满路。凤箫声动,\uline{\qquad\qquad\qquad\qquad},一夜鱼龙舞。

\question[2] 仍怜故乡水,\uline{\qquad\qquad\qquad\qquad}。

\question[2] 缺月挂疏桐,\uline{\qquad\qquad\qquad\qquad}。

\question[2] 有约不来过夜半,\uline{\qquad\qquad\qquad\qquad}。

\question[2] 众里寻他千百度,蓦然回首,\uline{\qquad\qquad\qquad\qquad},灯火阑珊处。

\question[2] \uline{\qquad\qquad\qquad\qquad},独上高楼,望尽天涯路。

\question[2] \uline{\qquad\qquad\qquad\qquad},见人惊起入芦花。

\end{questions}

\end{document}

