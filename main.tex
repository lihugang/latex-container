
\documentclass[12pt, a4paper, addpoints, answers]{exam}
\printanswers

\usepackage{xeCJK}
\setCJKmainfont{SimSun}[BoldFont=SimHei,ItalicFont=KaiTi]

\footer{}{第 \thepage 页 (共 \pageref{LastPage} 页)}{}
\firstpageheader{Test}{}{姓名:\qquad\qquad}
\runningheader{}{Test}{}

\usepackage{amssymb}
\usepackage{multicol}
\usepackage{lastpage}

\usepackage{xcolor}

\usepackage[
pdfa=true,
unicode=true,
hidelinks,
pdfauthor={李沪纲},
pdftitle={Test},
pdfsubject={Test}]{hyperref}

\usepackage{geometry}
\geometry{a4paper,scale=0.8}

\usepackage{ulem}

\pointformat{(\thepoints)}
\pointname{分}
\checkboxchar{$\square$}
\checkedchar{$\blacksquare$}

\setlength{\parindent}{2em}

\begin{document}

\pagestyle{headandfoot}

\begin{center}
\fbox{\fbox{\parbox{5.5in}{\centering
Test

命题人:李沪纲

(答题时间 2 分钟,共 \numquestions 题,满分 \numpoints 分)

于 2023/7/25 12:50:24 导出}}}
\end{center}
\vspace{5mm}

\normalsize
\vspace{5mm}

\section{\normalsize{填空题 (本大题共 299 题,满分 897 分)}}
\hspace{1.5cm}

\begin{questions}
\question[3] \fillin,绿柳才黄半未匀。

\question[3] 诗家清景在新春,\fillin。

\question[3] \fillin,出门俱是看花人。

\question[3] 若待上林花似锦,\fillin。

\question[3] \fillin,春日凝妆上翠楼。

\question[3] 闺中少妇不知愁,\fillin。

\question[3] \fillin,悔教夫婿觅封侯。

\question[3] 忽见陌头杨柳色,\fillin。

\question[3] \fillin,积雪浮云端。

\question[3] 终南阴岭秀,\fillin。

\question[3] \fillin,城中增暮寒。

\question[3] 林表明霁色,\fillin。

\question[3] \fillin?夹岸桃花蘸水开。

\question[3] 双飞燕子几时回?\fillin。

\question[3] \fillin,小舟撑出柳阴来。

\question[3] 春雨断桥人不度,\fillin。

\question[3] \fillin,青草池塘处处蛙。

\question[3] 黄梅时节家家雨,\fillin。

\question[3] \fillin,闲敲棋子落灯花。

\question[3] 有约不来过夜半,\fillin。

\question[3] \fillin,潮退渔船阁岸斜。

\question[3] 江头落日照平沙,\fillin。

\question[3] \fillin,见人惊起入芦花。

\question[3] 白鸟一双临水立,\fillin。

\question[3] \fillin,北风吹起数声雷。

\question[3] 海浪如云去却回,\fillin。

\question[3] \fillin,卧看千山急雨来。

\question[3] 朱楼四面钩疏箔,\fillin。

\question[3] \fillin,陂塘分出几泉清?

\question[3] 山郡逢春复乍晴,\fillin?

\question[3] \fillin,雪后千峰半入城。

\question[3] 郭边万户皆临水,\fillin。

\question[3] \fillin,来从楚国游。

\question[3] 渡远荆门外,\fillin。

\question[3] \fillin,江入大荒流。

\question[3] 山随平野尽,\fillin。

\question[3] \fillin,云生结海楼。

\question[3] 月下飞天镜,\fillin。

\question[3] \fillin,万里送行舟。

\question[3] 仍怜故乡水,\fillin。

\question[3] \fillin,秋水日潺湲。

\question[3] 寒山转苍翠,\fillin。

\question[3] \fillin,临风听暮蝉。

\question[3] 倚杖柴门外,\fillin。

\question[3] \fillin,墟里上孤烟。

\question[3] 渡头馀落日,\fillin。

\question[3] \fillin,狂歌五柳前。

\question[3] 复值接舆醉,\fillin。

\question[3] \fillin,涵虚混太清。

\question[3] 八月湖水平,\fillin。

\question[3] \fillin,波撼岳阳城。

\question[3] 气蒸云梦泽,\fillin。

\question[3] \fillin,端居耻圣明。

\question[3] 欲济无舟楫,\fillin。

\question[3] \fillin,徒有羡鱼情。

\question[3] 坐观垂钓者,\fillin。

\question[3] \fillin,连山接海隅。

\question[3] 太乙近天都,\fillin。

\question[3] \fillin,青霭入看无。

\question[3] 白云回望合,\fillin。

\question[3] \fillin,阴晴众壑殊。

\question[3] 分野中峰变,\fillin。

\question[3] \fillin,隔水问樵夫。

\question[3] 欲投人处宿,\fillin。

\question[3] \fillin?锦官城外柏森森。

\question[3] 丞相祠堂何处寻?\fillin。

\question[3] \fillin,隔叶黄鹂空好音。

\question[3] 映阶碧草自春色,\fillin。

\question[3] \fillin,两朝开济老臣心。

\question[3] 三顾频烦天下计,\fillin。

\question[3] \fillin,长使英雄泪满襟。

\question[3] 出师未捷身先死,\fillin。

\question[3] \fillin,往来成古今。

\question[3] 人事有代谢,\fillin。

\question[3] \fillin,我辈复登临。

\question[3] 江山留胜迹,\fillin。

\question[3] \fillin,天寒梦泽深。

\question[3] 水落鱼梁浅,\fillin。

\question[3] \fillin,读罢泪沾襟。

\question[3] 羊公碑尚在,\fillin。

\question[3] \fillin,徙倚欲何依。

\question[3] 东皋薄暮望,\fillin。

\question[3] \fillin,山山唯落晖。

\question[3] 树树皆秋色,\fillin。

\question[3] \fillin,猎马带禽归。

\question[3] 牧人驱犊返,\fillin。

\question[3] \fillin,长歌怀采薇。

\question[3] 相顾无相识,\fillin。

\question[3] \fillin,水寒风似刀。

\question[3] 饮马渡秋水,\fillin。

\question[3] \fillin,黯黯见临洮。

\question[3] 平沙日未没,\fillin。

\question[3] \fillin,咸言意气高。

\question[3] 昔日长城战,\fillin。

\question[3] \fillin,白骨乱蓬蒿。

\question[3] 黄尘足今古,\fillin。

\question[3] \fillin,金陵王气黯然收。

\question[3] 王濬楼船下益州,\fillin。

\question[3] \fillin,一片降幡出石头。

\question[3] 千寻铁锁沉江底,\fillin。

\question[3] \fillin,山形依旧枕寒流。

\question[3] 人世几回伤往事,\fillin。

\question[3] \fillin,故垒萧萧芦荻秋。

\question[3] 今逢四海为家日,\fillin。

\question[3] \fillin,信马悠悠野兴长。

\question[3] 马穿山径菊初黄,\fillin。

\question[3] \fillin,数峰无语立斜阳。

\question[3] 万壑有声含晚籁,\fillin。

\question[3] \fillin,荞麦花开白雪香。

\question[3] 棠梨叶落胭脂色,\fillin。

\question[3] \fillin,村桥原树似吾乡。

\question[3] 何事吟馀忽惆怅,\fillin。

\question[3] \fillin,更吹落、星如雨。

\question[3] 东风夜放花千树,\fillin。

\question[3] \fillin。凤箫声动,玉壶光转,一夜鱼龙舞。

\question[3] 宝马雕车香满路。\fillin,玉壶光转,一夜鱼龙舞。

\question[3] 宝马雕车香满路。凤箫声动,\fillin,一夜鱼龙舞。

\question[3] 宝马雕车香满路。凤箫声动,玉壶光转,\fillin。

\question[3] \fillin,笑语盈盈暗香去。

\question[3] 蛾儿雪柳黄金缕,\fillin。

\question[3] \fillin,蓦然回首,那人却在,灯火阑珊处。

\question[3] 众里寻他千百度,\fillin,那人却在,灯火阑珊处。

\question[3] 众里寻他千百度,蓦然回首,\fillin,灯火阑珊处。

\question[3] 众里寻他千百度,蓦然回首,那人却在,\fillin。

\question[3] \fillin,漏断人初静。

\question[3] 缺月挂疏桐,\fillin。

\question[3] \fillin,缥缈孤鸿影。

\question[3] 谁见幽人独往来,\fillin。

\question[3] \fillin,有恨无人省。

\question[3] 惊起却回头,\fillin。

\question[3] \fillin,寂寞沙洲冷。

\question[3] 拣尽寒枝不肯栖,\fillin。

\question[3] \fillin,罗幕轻寒,燕子双飞去。

\question[3] 槛菊愁烟兰泣露,\fillin,燕子双飞去。

\question[3] 槛菊愁烟兰泣露,罗幕轻寒,\fillin。

\question[3] \fillin,斜光到晓穿朱户。

\question[3] 明月不谙离恨苦,\fillin。

\question[3] \fillin,独上高楼,望尽天涯路。

\question[3] 昨夜西风凋碧树,\fillin,望尽天涯路。

\question[3] 昨夜西风凋碧树,独上高楼,\fillin。

\question[3] \fillin,山长水阔知何处?

\question[3] 欲寄彩笺兼尺素,\fillin?

\question[3] \fillin,午醉醒来愁未醒。送春春去几时回?

\question[3] 水调数声持酒听,\fillin。送春春去几时回?

\question[3] 水调数声持酒听,午醉醒来愁未醒。\fillin?

\question[3] \fillin,往事后期空记省。

\question[3] 临晚镜,伤流景,\fillin。

\question[3] \fillin,云破月来花弄影。

\question[3] 沙上并禽池上暝,\fillin。

\question[3] \fillin,风不定,人初静,明日落红应满径。

\question[3] 重重帘幕密遮灯,\fillin,明日落红应满径。

\question[3] 重重帘幕密遮灯,风不定,人初静,\fillin。

\question[3] \fillin,客来旋把朱帘挂。

\question[3] 低茅舍,卖酒家,\fillin。

\question[3] \fillin,方池睡鸭,老树昏鸦。

\question[3] 长天落霞,\fillin,老树昏鸦。

\question[3] 长天落霞,方池睡鸭,\fillin。

\question[3] \fillin,-幅王维画。

\question[3] 几句杜陵诗,\fillin。

\question[3] \fillin,相逢每醉还。

\question[3] 江汉曾为客,\fillin。

\question[3] \fillin,流水十年间。

\question[3] 浮云一别后,\fillin。

\question[3] \fillin,萧疏鬓已斑。

\question[3] 欢笑情如旧,\fillin。

\question[3] \fillin?淮上有秋山。

\question[3] 何因不归去?\fillin。

\question[3] \fillin,昨夜微霜初渡河。

\question[3] 朝闻游子唱离歌,\fillin。

\question[3] \fillin,云山况是客中过。

\question[3] 鸿雁不堪愁里听,\fillin。

\question[3] \fillin,御苑砧声向晚多。

\question[3] 关城曙色催寒近,\fillin。

\question[3] \fillin,空令岁月易蹉跎。

\question[3] 莫见长安行乐处,\fillin。

\question[3] \fillin,青天无片云。

\question[3] 牛渚西江夜,\fillin。

\question[3] \fillin,空忆谢将军。

\question[3] 登舟望秋月,\fillin。

\question[3] \fillin,斯人不可闻。

\question[3] 余亦能高咏,\fillin。

\question[3] \fillin,枫叶落纷纷。

\question[3] 明朝挂帆去,\fillin。

\question[3] \fillin,秋入望乡心。

\question[3] 古台摇落后,\fillin。

\question[3] \fillin,云峰隔水深。

\question[3] 野寺来人少,\fillin。

\question[3] \fillin,寒磬满空林。

\question[3] 夕阳依旧垒,\fillin。

\question[3] \fillin,长江独至今。

\question[3] 惆怅南朝事,\fillin。

\question[3] \fillin,黄昏饮马傍交河。

\question[3] 白日登山望烽火,\fillin。

\question[3] \fillin,公主琵琶幽怨多。

\question[3] 行人刁斗风沙暗,\fillin。

\question[3] \fillin,雨雪纷纷连大漠。

\question[3] 野云万里无城郭,\fillin。

\question[3] \fillin,胡儿眼泪双双落。

\question[3] 胡雁哀鸣夜夜飞,\fillin。

\question[3] \fillin,应将性命逐轻车。

\question[3] 闻道玉门犹被遮,\fillin。

\question[3] \fillin,空见蒲桃入汉家。

\question[3] 年年战骨埋荒外,\fillin。

\question[3] \fillin,祠荒向水开。

\question[3] 树老垂缨乱,\fillin。

\question[3] \fillin,古屋为风摧。

\question[3] 偶人经雨踣,\fillin。

\question[3] \fillin,渔郎奠竹杯。

\question[3] 野鸟栖尘坐,\fillin。

\question[3] \fillin,无奈《楚辞》哀!

\question[3] 欲传《山鬼》曲,\fillin!

\question[3] \fillin,长亭酒一瓢。

\question[3] 红叶晚萧萧,\fillin。

\question[3] \fillin,疏雨过中条。

\question[3] 残云归太华,\fillin。

\question[3] \fillin,河声入海遥。

\question[3] 树色随山迥,\fillin。

\question[3] \fillin,犹自梦渔樵。

\question[3] 帝乡明日到,\fillin。

\question[3] \fillin,栖栖翠幕中。

\question[3] 何处金衣客,\fillin。

\question[3] \fillin,无计啭春风。

\question[3] 有心惊晓梦,\fillin。

\question[3] \fillin,谁巢井上桐。

\question[3] 漫逐梁间燕,\fillin。

\question[3] \fillin,缄恨在雕笼。

\question[3] 空将云路翼,\fillin。

\question[3] \fillin,别离何易来何难。

\question[3] 岂有千山与万山,\fillin。

\question[3] \fillin,他乡故乡空倚阑。

\question[3] 一日一日似流水,\fillin。

\question[3] \fillin,松扶古殿三时寒。

\question[3] 云补断桥六月雨,\fillin。

\question[3] \fillin,几时猿鹤来同餐。

\question[3] 笋脯茶油新麦饭,\fillin。

\question[3] \fillin,松间沙路净无泥,萧萧暮雨子规啼。

\question[3] 山下兰芽短浸溪,\fillin,萧萧暮雨子规啼。

\question[3] 山下兰芽短浸溪,松间沙路净无泥,\fillin。

\question[3] \fillin?门前流水尚能西!休将白发唱黄鸡。

\question[3] 谁道人生无再少?\fillin!休将白发唱黄鸡。

\question[3] 谁道人生无再少?门前流水尚能西!\fillin。

\question[3] \fillin,万家杨柳青烟里。

\question[3] 小雨纤纤风细细,\fillin。

\question[3] \fillin,愁无比,和春付与东流水。

\question[3] 恋树湿花飞不起,\fillin,和春付与东流水。

\question[3] 恋树湿花飞不起,愁无比,\fillin。

\question[3] \fillin?金龟解尽留无计。

\question[3] 九十光阴能有几?\fillin。

\question[3] \fillin,拼一醉,而今乐事他年泪。

\question[3] 寄语东阳沽酒市,\fillin,而今乐事他年泪。

\question[3] 寄语东阳沽酒市,拼一醉,\fillin。

\question[3] \fillin,雁行初过客登楼。

\question[3] 霜后芙蓉落远洲,\fillin。

\question[3] \fillin,极目江南总是秋。

\question[3] 荒烟平楚苍茫处,\fillin。

\question[3] \fillin,天外三峰削不成。

\question[3] 岧峣太华俯咸京,\fillin。

\question[3] \fillin,仙人掌上雨初晴。

\question[3] 武帝祠前云欲散,\fillin。

\question[3] \fillin,驿路西连汉畤平。

\question[3] 河山北枕秦关险,\fillin。

\question[3] \fillin,何如此处学长生。

\question[3] 借问路旁名利客,\fillin。

\question[3] \fillin,何况悲笳出塞愁。

\question[3] 故乡飞鸟尚啁啾,\fillin。

\question[3] \fillin,翠崖遗迹为谁留。

\question[3] 青冢埋魂知不返,\fillin。

\question[3] \fillin,肉食何人与国谋。

\question[3] 玉颜自古为身累,\fillin。

\question[3] \fillin,岩花涧草自春秋。

\question[3] 行路至今空叹息,\fillin。

\question[3] \fillin,渐离击筑荆卿歌。

\question[3] 寒风夕吹易水波,\fillin。

\question[3] \fillin,日落登车去不顾。

\question[3] 白衣洒泪当祖路,\fillin。

\question[3] \fillin,舞阳色沮那敢呼。

\question[3] 秦王殿上开地图,\fillin。

\question[3] \fillin,事已不成空骂倨。

\question[3] 手持匕首摘铜柱,\fillin。

\question[3] \fillin! 燕丹寡谋当灭身。

\question[3] 吁嗟乎! \fillin。

\question[3] \fillin,惜哉枉杀樊将军。

\question[3] 田光自刎何足云,\fillin。

\question[3] \fillin,南望潇湘水一涯。

\question[3] 危楼樽酒赋蒹葭,\fillin。

\question[3] \fillin,岸枫遥映赤城霞。

\question[3] 云麓半涵青海雾,\fillin。

\question[3] \fillin,半缺河山待女娲。

\question[3] 双飞日月驱神骏,\fillin。

\question[3] \fillin,剑锋腾踏绕霜花。

\question[3] 学就屠龙空束手,\fillin。

\question[3] \fillin,独上高原愁日暮。

\question[3] 东风吹绽红亭树,\fillin。

\question[3] \fillin,吹作游人眼中雾。

\question[3] 可怜骊马啼下尘,\fillin。

\question[3] \fillin,荒园冻雀一声啼。

\question[3] 青山渐高日渐低,\fillin。

\question[3] \fillin,项羽坟头石马嘶。

\question[3] 三归台畔古碑没,\fillin。

\end{questions}

\hspace{5cm}

\section{\normalsize{填空题 (本大题共 117 题,满分 234 分)}}
\hspace{1.5cm}
\begin{multicols}{2}
\begin{questions}
\question[2] \fillin,君子以自强不息

\question[2] 天行健,\fillin

\question[2] \fillin,君子以厚德载物

\question[2] 地势坤,\fillin

\question[2] \fillin,匹夫不可夺志

\question[2] 三军可夺帅,\fillin

\question[2] \fillin,俯不怍于人

\question[2] 仰不愧于天,\fillin

\question[2] \fillin,是谓过也

\question[2] 过而不改,\fillin

\question[2] \fillin,终累大德

\question[2] 不矜细行,\fillin

\question[2] \fillin,可以明得失

\question[2] 以人为镜,\fillin

\question[2] \fillin,骤雨不终日

\question[2] 飘风不终朝,\fillin

\question[2] \fillin,非一木可支

\question[2] 大夏将倾,\fillin

\question[2] \fillin,美言不信

\question[2] 信言不美,\fillin

\question[2] \fillin,人老无刚

\question[2] 剑老无芒,\fillin

\question[2] \fillin,后事之师

\question[2] 前事不忘,\fillin

\question[2] \fillin,到老愈辣

\question[2] 姜桂之性,\fillin

\question[2] \fillin,则安之

\question[2] 既来之,\fillin

\question[2] \fillin,谬以千里

\question[2] 差之毫厘,\fillin

\question[2] \fillin偏逢连夜雨

\question[2] 屋漏\fillin连夜雨

\question[2] 屋漏偏逢\fillin

\question[2] \fillin,下自成蹊

\question[2] 桃李不言,\fillin

\question[2] \fillin满天下

\question[2] 桃李\fillin

\question[2] \fillin淡如水

\question[2] 君子之交\fillin

\question[2] \fillin甘若醴

\question[2] 小人之交\fillin

\question[2] \fillin到晓温

\question[2] 炉火灰深\fillin

\question[2] \fillin,疑义相与析

\question[2] 奇文共欣赏,\fillin

\question[2] \fillin,捻断数茎须

\question[2] 吟安一个字,\fillin

\question[2] \fillin万波随

\question[2] 一波才动\fillin

\question[2] \fillin万古新

\question[2] 一语天然\fillin

\question[2] \fillin,我脚高时他更高

\question[2] 一步一陟一回顾,\fillin

\question[2] \fillin,使我衣袖三年香

\question[2] 万人丛中一握手,\fillin

\question[2] \fillin,饥寒不累其心

\question[2] 毁誉不干其守,\fillin

\question[2] \fillin,专欲难成

\question[2] 众怒难犯,\fillin

\question[2] \fillin必从勤苦得

\question[2] 富贵\fillin勤苦得

\question[2] 富贵必从\fillin

\question[2] \fillin争何事

\question[2] 蜗牛角上\fillin

\question[2] \fillin,慎勿作桃李

\question[2] 愿君学长松,\fillin

\question[2] \fillin明镜台

\question[2] 心如\fillin

\question[2] \fillin,自有留人处

\question[2] 此处不留人,\fillin

\question[2] \fillin,小人长戚戚

\question[2] 君子坦荡荡,\fillin

\question[2] \fillin,以及人之老

\question[2] 老吾老,\fillin

\question[2] \fillin,不患无土

\question[2] 君子患无德,\fillin

\question[2] \fillin,更进一步

\question[2] 百尺竿头,\fillin

\question[2] \fillin,相逢何必曾相识

\question[2] 同是天涯沦落人,\fillin

\question[2] \fillin,同气相求

\question[2] 同声相应,\fillin

\question[2] \fillin,旁观者清

\question[2] 当局者迷,\fillin

\question[2] \fillin,岁岁年年人不同

\question[2] 年年岁岁花相似,\fillin

\question[2] \fillin而后私仇

\question[2] 先国家之急而\fillin

\question[2] \fillin,甚于防川

\question[2] 防民之口,\fillin

\question[2] \fillin,如见其人

\question[2] 如闻其声,\fillin

\question[2] \fillin一家

\question[2] 自成\fillin

\question[2] \fillin,禀性难移

\question[2] 江山易改,\fillin

\question[2] \fillin不及掩耳

\question[2] 迅雷\fillin掩耳

\question[2] 迅雷不及\fillin

\question[2] \fillin空手回

\question[2] 如入宝山\fillin

\question[2] \fillin非礼也

\question[2] 来而不往\fillin

\question[2] \fillin半九十

\question[2] \fillin如珠

\question[2] 好语\fillin

\question[2] \fillin救不得远火

\question[2] 近水\fillin远火

\question[2] 近水救不得\fillin

\question[2] \fillin,退可以守

\question[2] 进可以攻,\fillin

\question[2] \fillin,逸于作文

\question[2] 劳于读书,\fillin

\question[2] \fillin图治

\question[2] 励精\fillin

\question[2] \fillin泰来

\question[2] 否极\fillin

\end{questions}
\end{multicols}

\hspace{5cm}

\section{\normalsize{多选题 (本大题共 4 题,满分 20 分)}}
\hspace{1.5cm}

\begin{questions}
\question[5] AB

\begin{oneparcheckboxes}
\choice  1
\choice  2
\choice  3
\choice  4
\end{oneparcheckboxes}

\question[5] ABCD

\begin{oneparcheckboxes}
\choice  1
\choice  2
\choice  3
\choice  4
\end{oneparcheckboxes}

\question[5] C

\begin{oneparcheckboxes}
\choice  1
\choice  2
\choice  3
\choice  4
\end{oneparcheckboxes}

\question[5] none

\begin{oneparcheckboxes}
\choice  1
\choice  2
\choice  3
\choice  4
\end{oneparcheckboxes}

\end{questions}

\hspace{5cm}

\section{\normalsize{单选题 (本大题共 1 题,满分 20 分)}}
\hspace{1.5cm}

\begin{questions}
\question[20] PigD

\begin{oneparchoices}
\choice  2
\choice  1
\choice  3
\choice  4
\end{oneparchoices}

\end{questions}

\hspace{5cm}

\section{\normalsize{主观题 (本大题共 1 题,满分 30 分)}}
\hspace{1.5cm}

\begin{questions}
\question[30] 赏析“可怜无定河边骨,犹是春闺梦里人”

\end{questions}

\end{document}

