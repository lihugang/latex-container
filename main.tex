
\documentclass[12pt, a4paper, addpoints]{exam}
\usepackage{xeCJK}
\setCJKmainfont{SimSun}[BoldFont=SimHei,ItalicFont=KaiTi]

\footer{}{第 \thepage 页 (共 \pageref{LastPage} 页)}{}
\firstpageheader{Test}{}{\textbf{标准答案}}
\runningheader{}{Test}{\textbf{标准答案}}

\usepackage{amssymb}
\usepackage{multicol}
\usepackage{lastpage}

\usepackage{xcolor}

\usepackage[
pdfa=true,
unicode=true,
hidelinks,
pdfauthor={李沪纲},
pdftitle={Test},
pdfsubject={Test}]{hyperref}

\usepackage{geometry}
\geometry{a4paper,scale=0.8}

\usepackage{ulem}

\pointformat{(\thepoints)}
\pointname{分}
\checkboxchar{$\square$}
\checkedchar{$\blacksquare$}

\setlength{\parindent}{2em}

\begin{document}

\pagestyle{headandfoot}

\begin{center}
\fbox{\fbox{\parbox{5.5in}{\centering
Test

\textbf{标准答案}

命题人:李沪纲

(答题时间 2 分钟,共 \numquestions 题,满分 \numpoints 分)

于 2023/7/25 12:37:57 导出}}}
\end{center}
\vspace{5mm}

\normalsize
\vspace{5mm}

\section{\normalsize{填空题 (本大题共 20 题,满分 60 分)}}
\hspace{1.5cm}

\begin{questions}
\question[3] 山下兰芽短浸溪,松间沙路净无泥,萧萧暮雨子规啼。

\question[3] 浮云一别后,流水十年间。

\question[3] 玉颜自古为身累,肉食何人与国谋。

\question[3] 黄梅时节家家雨,青草池塘处处蛙。

\question[3] 青冢埋魂知不返,翠崖遗迹为谁留。

\question[3] 众里寻他千百度,蓦然回首,那人却在,灯火阑珊处。

\question[3] 鸿雁不堪愁里听,云山况是客中过。

\question[3] 东风吹绽红亭树,独上高原愁日暮。

\question[3] 气蒸云梦泽,波撼岳阳城。

\question[3] 重重帘幕密遮灯,风不定,人初静,明日落红应满径。

\question[3] 千寻铁锁沉江底,一片降幡出石头。

\question[3] 云麓半涵青海雾,岸枫遥映赤城霞。

\question[3] 黄尘足今古,白骨乱蓬蒿。

\question[3] 白日登山望烽火,黄昏饮马傍交河。

\question[3] 海浪如云去却回,北风吹起数声雷。

\question[3] 长天落霞,方池睡鸭,老树昏鸦。

\question[3] 蛾儿雪柳黄金缕,笑语盈盈暗香去。

\question[3] 若待上林花似锦,出门俱是看花人。

\question[3] 重重帘幕密遮灯,风不定,人初静,明日落红应满径。

\question[3] 出师未捷身先死,长使英雄泪满襟。

\end{questions}

\hspace{5cm}

\section{\normalsize{填空题 (本大题共 10 题,满分 20 分)}}
\hspace{1.5cm}
\begin{multicols}{2}
\begin{questions}
\question[2] 毁誉不干其守,饥寒不累其心

\question[2] 当局者迷,旁观者清

\question[2] 过而不改,是谓过也

\question[2] 既来之,则安之

\question[2] 如闻其声,如见其人

\question[2] 一波才动万波随

\question[2] 此处不留人,自有留人处

\question[2] 防民之口,甚于防川

\question[2] 否极泰来

\question[2] 剑老无芒,人老无刚

\end{questions}
\end{multicols}

\hspace{5cm}

\section{\normalsize{多选题 (本大题共 4 题,满分 20 分)}}
\hspace{1.5cm}

\begin{questions}
\question[5] none

\begin{oneparcheckboxes}
\choice  1
\choice  2
\choice  3
\choice  4
\end{oneparcheckboxes}

\question[5] C

\begin{oneparcheckboxes}
\choice  1
\choice  2
\CorrectChoice 3
\choice  4
\end{oneparcheckboxes}

\question[5] AB

\begin{oneparcheckboxes}
\CorrectChoice 1
\CorrectChoice 2
\choice  3
\choice  4
\end{oneparcheckboxes}

\question[5] ABCD

\begin{oneparcheckboxes}
\CorrectChoice 1
\CorrectChoice 2
\CorrectChoice 3
\CorrectChoice 4
\end{oneparcheckboxes}

\end{questions}

\hspace{5cm}

\section{\normalsize{单选题 (本大题共 1 题,满分 20 分)}}
\hspace{1.5cm}

\begin{questions}
\question[20] PigD

\begin{oneparchoices}
\choice  2
\choice  1
\choice  3
\CorrectChoice 4
\end{oneparchoices}

\end{questions}

\hspace{5cm}

\section{\normalsize{主观题 (本大题共 1 题,满分 30 分)}}
\hspace{1.5cm}

\begin{questions}
\question[30] 赏析“可怜无定河边骨,犹是春闺梦里人”

参考答案: \textbf{此句没有直接战争带来的悲惨景象,也没有渲染家人的悲伤情绪,而是匠心独运,把“河边骨”和“春闺梦”联系起来,写闺中妻子不知征人战死,仍然在梦中想见已成白骨的丈夫,使全诗产生震撼心灵的悲剧力量。}

\end{questions}

\end{document}

