
\documentclass[12pt, a4paper, addpoints]{exam}
\usepackage{xeCJK}
\setCJKmainfont { SimSun } [ BoldFont = SimHei , ItalicFont = KaiTi ]

\footer{}{第 \thepage 页 (共 \pageref{LastPage} 页)}{}
\firstpageheader{诗句默写 / 2023古诗文}{}{姓名:\qquad\qquad}
\runningheader{}{诗句默写 / 2023古诗文}{}

\usepackage{lastpage}

\usepackage[
pdfa=true,
unicode=true,
hidelinks,
pdfauthor={李沪纲},
pdftitle={诗句默写 / 2023古诗文},
pdfsubject={诗句默写 / 2023古诗文}]{hyperref}

\usepackage{geometry}
\geometry{a4paper,scale=0.8}

\usepackage{ulem}

% \usepackage{fancyhdr}
% \pagestyle{fancy}
% \fancyhead[L]{诗句默写 / 2023古诗文}
% \fancyhead[R]{班级:\qquad 姓名:\qquad\qquad}
% \fancyfoot[C]{第 \thepage 页 (共 \pageref{LastPage} 页)}

\pointformat{(\thepoints)}
\pointname{分}

\setlength{\parindent}{2em}

\begin{document}

\pagestyle{headandfoot}

\begin{center}
\fbox{\fbox{\parbox{5.5in}{\centering
诗句默写 / 2023古诗文

命题人:李沪纲

(答题时间 10 分钟,共 50 题,满分 100 分)

于 2023/7/17 10:26:21 导出}}}
\end{center}
\vspace{5mm}

\normalsize
\vspace{5mm}

\section{\normalsize{填空题 (本大题共 50 题,满分 100 分)}}
\hspace{1.5cm}
\begin{questions}
\question[2] 渡头馀落日,\uline{\qquad\qquad\qquad\qquad}。

\question[2] \uline{\qquad\qquad\qquad\qquad},独上高楼,望尽天涯路。

\question[2] \uline{\qquad\qquad\qquad\qquad},临风听暮蝉。

\question[2] 若待上林花似锦,\uline{\qquad\qquad\qquad\qquad}。

\question[2] \uline{\qquad\qquad\qquad\qquad},阴晴众壑殊。

\question[2] 江山留胜迹,\uline{\qquad\qquad\qquad\qquad}。

\question[2] \uline{\qquad\qquad\qquad\qquad},闲敲棋子落灯花。

\question[2] \uline{\qquad\qquad\qquad\qquad},更吹落、星如雨。

\question[2] 平沙日未没,\uline{\qquad\qquad\qquad\qquad}。

\question[2] \uline{\qquad\qquad\qquad\qquad},天寒梦泽深。

\question[2] \uline{\qquad\qquad\qquad\qquad},隔叶黄鹂空好音。

\question[2] 林表明霁色,\uline{\qquad\qquad\qquad\qquad}。

\question[2] \uline{\qquad\qquad\qquad\qquad},涵虚混太清。

\question[2] \uline{\qquad\qquad\qquad\qquad},端居耻圣明。

\question[2] 春雨断桥人不度,\uline{\qquad\qquad\qquad\qquad}。

\question[2] \uline{\qquad\qquad\qquad\qquad},往来成古今。

\question[2] \uline{\qquad\qquad\qquad\qquad},有恨无人省。

\question[2] \uline{\qquad\qquad\qquad\qquad},秋水日潺湲。

\question[2] 昨夜西风凋碧树,\uline{\qquad\qquad\qquad\qquad},望尽天涯路。

\question[2] 白云回望合,\uline{\qquad\qquad\qquad\qquad}。

\question[2] 三顾频烦天下计,\uline{\qquad\qquad\qquad\qquad}。

\question[2] 槛菊愁烟兰泣露,\uline{\qquad\qquad\qquad\qquad},燕子双飞去。

\question[2] 宝马雕车香满路。凤箫声动,\uline{\qquad\qquad\qquad\qquad},一夜鱼龙舞。

\question[2] \uline{\qquad\qquad\qquad\qquad},北风吹起数声雷。

\question[2] \uline{\qquad\qquad\qquad\qquad},蓦然回首,那人却在,灯火阑珊处。

\question[2] 闺中少妇不知愁,\uline{\qquad\qquad\qquad\qquad}。

\question[2] 丞相祠堂何处寻?\uline{\qquad\qquad\qquad\qquad}。

\question[2] 惊起却回头,\uline{\qquad\qquad\qquad\qquad}。

\question[2] 太乙近天都,\uline{\qquad\qquad\qquad\qquad}。

\question[2] 众里寻他千百度,蓦然回首,那人却在,\uline{\qquad\qquad\qquad\qquad}。

\question[2] 昨夜西风凋碧树,独上高楼,\uline{\qquad\qquad\qquad\qquad}。

\question[2] \uline{\qquad\qquad\qquad\qquad},狂歌五柳前。

\question[2] 王濬楼船下益州,\uline{\qquad\qquad\qquad\qquad}。

\question[2] \uline{\qquad\qquad\qquad\qquad},潮退渔船阁岸斜。

\question[2] \uline{\qquad\qquad\qquad\qquad}?夹岸桃花蘸水开。

\question[2] \uline{\qquad\qquad\qquad\qquad},金陵王气黯然收。

\question[2] 黄尘足今古,\uline{\qquad\qquad\qquad\qquad}。

\question[2] 江头落日照平沙,\uline{\qquad\qquad\qquad\qquad}。

\question[2] \uline{\qquad\qquad\qquad\qquad}?锦官城外柏森森。

\question[2] 复值接舆醉,\uline{\qquad\qquad\qquad\qquad}。

\question[2] 千寻铁锁沉江底,\uline{\qquad\qquad\qquad\qquad}。

\question[2] \uline{\qquad\qquad\qquad\qquad},罗幕轻寒,燕子双飞去。

\question[2] 何事吟馀忽惆怅,\uline{\qquad\qquad\qquad\qquad}。

\question[2] \uline{\qquad\qquad\qquad\qquad},青霭入看无。

\question[2] 分野中峰变,\uline{\qquad\qquad\qquad\qquad}。

\question[2] 众里寻他千百度,蓦然回首,\uline{\qquad\qquad\qquad\qquad},灯火阑珊处。

\question[2] \uline{\qquad\qquad\qquad\qquad},雪后千峰半入城。

\question[2] 白鸟一双临水立,\uline{\qquad\qquad\qquad\qquad}。

\question[2] 欲济无舟楫,\uline{\qquad\qquad\qquad\qquad}。

\question[2] \uline{\qquad\qquad\qquad\qquad},万里送行舟。

\end{questions}

\end{document}

